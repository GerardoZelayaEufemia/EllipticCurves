\documentclass{article}
\title{Zelaya All NT I know}
\author{Zelaya Eufemia}
\date{}
\usepackage{multicol}
\usepackage[margin=0.5in]{geometry}
\usepackage{amssymb}
\usepackage{amsmath}
\usepackage{multicol}
\setlength{\columnsep}{0.5cm}

\newcommand{\Q}{\mathbb{Q}}
\newcommand{\R}{\mathbb{R}}
\newcommand{\C}{\mathbb{C}}
\newcommand{\Z}{\mathbb{Z}}
\newcommand{\F}{\mathbb{F}}
\newcommand{\HH}{\mathbb{H}}
\newcommand{\hh}{\mathfrak{h}}
\newcommand{\PP}{\mathbb{P}}
\newcommand{\N}{\mathbb{N}}
\newcommand{\D}{\mathbb{D}}
\newcommand{\T}{\mathbb{T}}
\newcommand{\M}{\mathbb{M}}
\newcommand{\B}{\mathbb{B}}


\newcommand{\DD}{\mathcal{D}}
\newcommand{\EE}{\mathcal{E}}

\newcommand{\I}{\mathbb{I}}
\newcommand{\OO}{\mathcal{O}}
\newcommand{\af}{\mathfrak{a}}
\newcommand{\pp}{\mathfrak{p}}
\newcommand{\qq}{\mathfrak{q}}
\newcommand{\rr}{\mathfrak{r}}
\newcommand{\ff}{\mathfrak{f}}

\newcommand{\mm}{\mathfrak{m}}
\newcommand{\nn}{\mathfrak{n}}

\newcommand{\wf}{\mathcal{P}}
\newcommand{\FF}{\mathcal{F}}
\newcommand{\GG}{\mathcal{G}}



\newcommand{\ra}{\rightarrow}
\newcommand{\la}{\leftarrow}
\newcommand{\Ra}{\Rightarrow}
\newcommand{\La}{\Leftarrow}
\newcommand{\lra}{\leftrightarrow}
\newcommand{\Lra}{\Leftrightarrow}

\newcommand{\da}{\downarrow}
\newcommand{\ua}{\uparrow}
\newcommand{\raa}[1]{\overset{#1}{\longrightarrow}}
\newcommand{\laa}[1]{\overset{#1}{\longleftarrow}}
\newcommand{\lraa}[1]{\overset{#1}{\longleftrightarrow}}
\newcommand{\hra}{\hookrightarrow}
\newcommand{\bmath}[1]{\boldsymbol{#1}}
\newcommand{\limra}[1]{\lim_{\overset{\rightarrow}{#1}}}
\newcommand{\limla}[1]{\lim_{\overset{\leftarrow}{#1}}}
\newcommand{\ACK}{\overline{K}}
\newcommand{\AS}{\mathbb{A}^n}
\newcommand{\PS}{\mathbb{P}^n}
\begin{document}
\maketitle
\section{Computational CFT}
\begin{multicols}{2}
\begin{enumerate}
\item Ramified primes $p = \prod \pp^e$ and $\exists e > 1$. Or Real prime extending to complex (non-real) prime. 

\item \textbf{Modulus} $m = m_0 m_\infty$. And $(\OO_K/m)^* = (\OO_K/m_0)^* \times \FF_2^{m_\infty}$. Notation: $\alpha \equiv 1 (*\mm)$ if $\alpha \equiv 1 (\mm_0)$ and $\forall \sigma \in \mm_\infty: \sigma(\alpha) > 0$. 

\item \textbf{Euler Function} for Modulus: 
\[\phi(\mm) = |(\OO_K/\mm)^*| = 2^{|\mm_\infty|}\prod N(\pp)^{a-1}(N(\pp) - 1)\]

\item \textbf{Congruence Group:} $P_m \subseteq C \subseteq I_m$ where $P_m$ is the principal ideals generated by $\equiv 1 (*m)$ and $I_m$ is fractional ideals coprime with $m_0$. Denoted $(m,C)$.

\item \textbf{Equivalence Congruence Subgroups}: 
\[m|n \Ra (C,n) \sim (CP_m,m) \La I_n \cap P_m \subseteq C\]
\[m|n \Ra (C,m) \sim (C \cap I_n, n)\]

\item \textbf{Conductor}: Smallest $f$ such that $(C_f,f)$ is defined.\\
$cond(C_f,f) = f$ and $P_f \subseteq D \subseteq C_f\subseteq I_f \Ra cond(D,f) = f$. Makes it a conductor if $cond(P_f,f) = f$. \\
Conductor if $n|f$ and $n \neq f \Ra h_n < h_f$. 

\item \textbf{Characters}: \\
$(\chi,m)$ if $\chi:I_m \ra \C^*$ and $P_m \subseteq Ker(\chi) \subseteq I_m$.\\
$(\chi,m, C)$ if also $P_m \subseteq C \subseteq Ker(\chi) \subseteq I_m$.\\
\textbf{Primitive}: $cond(Ker(\chi,m),m) = m$.

\item
\[(C,m) \Ra C = \bigcap_{(\chi,m,C)}Ker(\chi)\]
\[cond(C,m) = LCM(cond(\chi,m,C))\]

\item Conductors cannot violate certain criteria P3.3.18(147) and P.3.3.19(148).

\item \textbf{Conductor of Abelian Extensions $L/K$}: For each prime $p$ there is a minimal integer $k_p \geq 0$ such that $x \equiv 1 (*p^k)$ is a local norm modulo $p$, i.e.: $\forall n \geq 0: x \equiv N_{L/K}(y_n) (*p^n)$. 
\[cond(L/K) = \left(\prod_p p^{k_p}\right) f_\infty(L/K)\]
where the $\infty$-part are the ramified real primes. 

\item \textbf{ANT}: For $\pp|p$: $D(\pp)/I(\pp) \cong Gal(\OO_L/\pp: \OO_K/p)$ canonically. For abelian extensions and unramified primes $I(\pp) = \{1_{Gal(L/K)}\}$ and thus the Frobenius $\forall x \in \OO_L: \sigma_\pp(x) \equiv x^{N(p)} (\pp)$.

\item \textbf{Suitable modulus} for $L/K$: $cond(L/K)|m$. Then \textbf{Artin Map} is a surjective grp hom on $I_m$ for suitable modulus $m$ with kernel $A_m(L/K)$ a congruence subgroup mod $m$: 
\[Art_{L/K}(a) = (\frac{L/K}{a}) = \prod_{p|a} Frob_p^{v_p(a)}\]

\item Computable \textbf{Norm Group or Takagi Group}: $A_m(L/K) = T_m(L/K) = P_m N_{L/K}(I_{m,L})$ (recall that $m$ is modulus in $K$). 
\[f = \min\{p^f \in T_m(L/K)\} = f(\pp|p)\]
\[T_m(L/K) = \langle p^f - N_{L/K}(\pp)\]
\[cond(m, A_m(L/K)) = cond(L/K)\]



\item \textbf{Hilbert CF}: Maximal Unramified Abelian $K_(1)$. Decomposition: $\min f/\pp^f = (\rho) \in Cl_K \Ra \pp = \prod^{h_K/f} \qq$. All ideals in $K$ become principal in $K(1)$, yet $h_{K(1)} \neq 1$ in general. 


\item \textbf{Ray Class Field}: $K(\mm)$ maximal abelian extension un-ramified outside $\mm_0$ and ramified at primes dividing $\mm_0$. 
\[G(K(\mm)/K) \cong Cl_\mm\]
\[\exists L = K(\mm)^{\overline{C}} \lra \overline{C} = C/P_\mm\]  
\[G(K(\mm)/L) \cong \overline{C} = C/P_\mm\]
\[G(L/K) \cong I_\mm/C\]
\[h_{\mm,C} = |I_\mm/C| = |Cl_\mm/\overline{C}|\]

\item Artin Reciprocity is Surjective Homomorphism $Cl_\mm \ra Gal(L/K)$ where $\mm$ divides the conductor $\ff$ of $L/K$. The Kernel is $A_\mm(L/K)$ a congruence subgroup. 

\item For a congruence subgroup $(\mm,C)$ there is an abelian $L/K$ and a $\mm|\ff$ with $C = A_\mm(L/K)$. i.e.: bijection between abelian extensions and congrouence subgroups classes. 
 
 
 

\end{enumerate}
\end{multicols}

\section{Abelian Varieties}
\begin{multicols}{2}
\begin{enumerate}
\item \textbf{Abelian Varieties}: Complete, actually projective, and connected Group Varieties, thus non-singular, irreducible. They are polarizable Torus or have a Riemann Form. Always Semisimple. 

\item \textbf{Morphisms of AV}: Regular Maps, homomorphisms followed by translations. 

\item \textbf{Riemann Form}: $E: L \times L \ra \Z$  $\R$-skew symmetric, extended to $E_\R: V \times V \ra \R$, $H_\R = E(i-,-) + iE$ Hermitian is positive definite. 

\item \textbf{Cohomology}:
\begin{itemize}
\item \textbf{Cup Product} $H^*(X,\Z) = \bigoplus H^r(X,\Z)$ ring with $a_r \cup b_s = (-1)^{rs} b^s \cup a^r$.
\item \textbf{Krummer} $H^m(X \times Y,\Z) = \bigoplus_{r+s = m}H^r(X,\Z) \otimes H^s(Y,\Z)$ and $a \otimes b \mapsto p_X^*a \otimes p_Y^*b$.
\item $H^1(X,\Z) \cong Hom(\pi_1(X,x) \ra \Z)$.
\item For Compact Orientable: $H^r \cong H_{d-r} \cong \hat{H}^{d-r}$
\item \[t \mapsto \chi(\mathcal{F}_t) = \sum_r (-1)^r dim_{k(t)} H^r(V_t,\mathbb{F}_t) \]
\end{itemize}

\item \textbf{Cohomology of the Torus} $X = V/L \cong \mathbb{T}^{2g}$ and 
\[H^r(\T^{2g},\Z) = \Z^{\binom{2g}{r}}\]
Canonical 
\[\Lambda^rH^1(X,\Z) \cong H^r(X,\Z) \cong Hom(\Lambda^r L, \Z)\]


\item 
\[\Lambda^r M = \frac{\bigoplus^r M}{\langle \otimes a_i/ a_i = a_j \rangle}\]
\[Hom(\Lambda^rM,\Z) = \{\textbf{Alt form } f: M^r \ra \Z\}\]

 
\item \textbf{Dual Torus}: 
\[\frac{V^* = \{f: V \ra \C \textrm{ sequilinear}\}} {L^* = \{f \in V^*/Im(f) \subseteq \Z\}}\]


\item \textbf{Weil Pairing} $X_m = X[m]$
\[X_m \times \hat{X}_m \ra \mu_m \cong \Z/m\Z\] 

\item \textbf{Isogeny from Riemann Form}: $\lambda_E: V \ra V^*/ v \mapsto H(v,-)$. \\
\textbf{Principal}: $|ker| = 1$



\item \textbf{Notation}: $\alpha: V \ra Spec(k)$. Coherent sheaves on $Spec(k)$ are finite dimensional vector spaces. \\
For $\OO_V$-module sheaf $M$: $\alpha_*M = \Gamma(V,M)$.\\
For a vector space $M = \bigoplus ke_i$: $\alpha^*M = \OO_V \otimes_k M = \bigoplus \OO_V e_i$.  

\item \textbf{Complete Linear System}: 
meromorphic functions $L(D) = \{f/ D + div(f) \geq 0\} \lra$ divisors $L(D) = \{D' = D + div(f) \geq D, \sim D\}$.  
\textbf{Linear Systems}: $W \subseteq L(D_0) \lra \{D_0 + div(f)/ f \in W\}$. 

\item \textbf{Fixed Divisor} of a Linear System: $E/ \forall D' \in L: D' \geq E$. \textbf{Base point} of a Linear System: $\forall D \in L: P \in Supp(D)$.

\item \textbf{Separate Points}: $P,Q \in V \Ra \exists D \in L: P \in Supp(D)$ and $Q \notin Supp(D)$. \textbf{Separate Tangent Directions}: $P \in v$ and $t \in T_P(V) \Ra \exists D \in L: P \in D$ and $t \notin Tgt_P(D)$. 

\item \textbf{Linear System Map}: $A \ra \PP^n/P \mapsto (f_i(P))$. Defined at all non-base points. If no base points, closed immersion iff separate points and tangent directions. 

\item \textbf{Very Ample Divisor}: $D/L(D)$ provides a closed immersion of $V \hra \PP^n$. \textbf{Ample Divisor}: $nD$ very ample. Known: $D$ ample $\Ra 3D$ very ample. For EC: ample if $deg(D) > 0$ and very ample if $deg(D) > 3 = 2g + 1$. 

\item \textbf{Isogenies}: Surjective homomorphism with finite kernel. Or finite, flat and surjective. \textbf{Degree} of an isogeny: degree as regular map, $[k(A):\alpha^*k(B)]$. \\
Separable $\Ra$ Etale $\Ra$ Every fibre has exactly $deg(\alpha)$ points. 



\item \textbf{Dual AV}: For $L$ invertible: 
Construction: $G$ finite acting on $V$ by regular maps (on the right), with orbits contained in open affine. Then $\pi: V \ra W = V/G$ is a finite regular map and a topological quotient .

\[\lambda_L:A(k) \ra Pic(A)/ a \mapsto t_a^* L \otimes L^{-1}\]
\[K(L) = \{a \in A: \lambda_L(a) = 0\}\]
$L$ invertible, $\Gamma(A,L) \neq 0$: $L$ ample iff $dim\,K(L) = 0$.
$K(L) = A$ iff $\forall a \in A(k^{al}): t_a^*L \cong L$.

\item The elements of $\hat{A}$ should parametrize $Pic^0(A)$: The \textbf{Poincare Sheaf} $P$ on $A \times \hat{A}$ has (1) $P|_{A \times \{b\}} \in Pic^0(A_b)$; $P|_{\{0\} \times \hat{A}}$ trivial. The Universal Property is that if $(T,L)$ satisfies similar properties than $(A,P)$ then there is a unique regular map $\alpha: T \ra A$ with $(1 \times \alpha)^* P \cong L$. 
\[\hat{A}(k) = Pic^0(A)\]

\item If $L$ is a divisorial correspondence between $A$ and $B$, meaning an invertible sheaf on $A \times B$ with trivial restriction to $A \times 0$ and $0 \times B$; and $s$ be the switching map $(a,b) \ra (b,a)$. Then (1) $(B,L)$ is the dual of $A$ iff (2) $L|_{A \times \{b\}}$ trivial $\Ra b = 0$ iff (3) $L|_{\{a\} \times B}$ trivial $\Ra a = 0$ iff (4) $(A,s^*L)$ is the dual of $B$. 

\item $G$ acting freely on $V$ and $W = V/G$. A coherent $G$-sheaf on $V$ is coherent $M$ of $\OO_V$ modules, with an action of $G$ on $M$ compatible with tis action on $V$. 

\item Under the two above $M \ra \pi^*M$ is an equivalence between coherent $\OO_V$-mods and coherent $G$-sheaves on $V$ respecting locally free of rank $r$ sheaves. 

\item $L$ ample means $\lambda_L: A \ra Pic^0(A)$ is surjective. 

\section{Endomorphisms}
\item Semisimplicity on AV: 
\[A \sim \prod A_i^{n_i}\]
 where $A_i$ are simple and pairwise non-isogenous. 

\item $End^0(A) = End(A) \otimes \Q \cong \prod M_{n_i}(D_i)$ where $D_i$ are division algebras $End^0(A_i)$.  
 
\item 
\[A(k^{sep})_{tors} = (\Q/\Z)^{2g}\]
\[A(k^{sep})(l) = (\Q_l/\Z_l)^{2g}\]
Recall $\Lambda = H_1(A,\Z) = \Z^{2g}$.
\[T_l(A) = \Z_l^{2g} = H_1(A,\Z) \otimes \Z_l\]
Also the first etale homology group of $A$. 

\item $Hom(A,B) \hra Hom_{\Z_l}(T_lA, T_lB)$.

\item $\alpha \in End(A) \Ra \hat{\alpha} \in End(H_1(A,\Q) = \Q^{2g})$ and $P_\alpha(X) = \det(\alpha - X|H_1(A,\Q))$, which is monic, degree $2g$, coefficients in $\Z$. Yet this does not generalize. 

\item $\forall \alpha \in End(A): \exists ! P_\alpha \in \Z[x]$ monic degree $2g$/ $\forall r \in \Z: P_\alpha(r) = deg(\alpha - r)$. This generalize to $\alpha \in End^0(A)$.

\item Characteristic Polynomial:  
\[P_\alpha(X) = n^{-2g}P_{n\alpha}(nX)\]
where $\alpha \in End^0(A), n \in \Z, n\alpha \in End(A)$. In particular $\forall r \in \Q: P_\alpha(r) = deg(\alpha - r)$. The degree function on $End^0(A)$ is a homogeneous polynomial function of degree $2g$, i.e.: $f(\sum k_ie_i)$ is a polynomial in the $k_i$'s.
\[P_\alpha(X) = X^{2g} - Tr(\alpha)X^{2g-1} + \ldots + deg(\alpha)\]


\item \textbf{Central Simple Division Algebra}: Central has $k$ in its center, Simple is finite dimensional and simple ring, Division is equal to its fractional ring. 

\item \textbf{Brauer Group of a Field}: $Br(k)$ Isomorphic classes of central (simple) division algebras. The product is the tensor product of representative of classes. The inverse is $D^{op}$. 

\item For local fields $inv: Br(k) \hra \Q/\Z$. If $k$ is non-archimedean, $inv$ is an isomorphism; if $k = \R$, then image is $(1/2)\Z/\Z$; and if $k$ is algebraically closed, then $Br(k) = 0$.

\item For Number Fields:
\[0 \ra Br(k) \ra \bigoplus_v Br(k_v) \ra \Q/\Z \ra 0\]
 

\end{enumerate}
\end{multicols}




\section{Elliptic Curves}
\begin{multicols}{2}
\begin{enumerate}
\item \textbf{Gauss Arithmetic Geometric Mean}: $(a_n,b_n) = (\frac{a_{n-1}+b_{n-1}}{2}, \sqrt{a_{n-1}b_{n-1}})$ converges fast.

\item \textbf{Big O - Small O}: For $f,g: S \ra \R$, $f = g + O(1)$ means $f-g \in [C_1,C_2]$ on $S$; $f \geq g + O(1)$ means $f-g \geq C_1$ on $S$; $f \leq g + O(1)$ means $f - g \leq C_2$ on $S$.

\item \[G(\Q^{Ab}/\Q) \cong \prod_p \Z_p^*\]

\item $1$-dimensional algebraic curves of genus $1$. Elliptic Curves equation and the relation between $y^2,xy,y,x^3,x^2,x,1$ can be obtained either by the Riemann Roch theorem on curves, or by the fact that Curves correspond to finite extensions of $K(t)$, then $K(t,s)/f(s,t)$

\item \textbf{Field of definition} $E/F$: Smallest $F/ a_i \in F$. 

\item \textbf{Abelian Group Structure}: Tangents-Secants $a + b + c = 0$.

\item \textbf{Weierstrass Equation}: $E = W(a_i)$.
\[y^2 + a_1xy + a_3y = x^3 + a_2x^2 + a_4x + a_6\]

\item \textbf{$\C$-Isomorphism homomorphism changes}: $x = u^2x' + r, y = u^3y' + u^2sx' + t; 0 \neq u,r,s,t \in \overline{K}$
\[a_1'= (2s+a_1)u^{-1}; a_2'= (3r + a_2 -s^2 - a_1s)u^{-2} \]
\[a_3'= (2t + a_1r + a_3)u^{-3} \]
\[a_4'= (3r^2 + 2a_2r + a_4 -2st-a_1t - a_1rs - a_3s)u^{-4} \]
\[a_6'=(r^3 + a_2r^2 + a_4r + a_6 -t^2 -a_1rt- a_3t)u^{-6} \]
\[\Delta' = u^{-12}\Delta; c_4' = c_4u^{-4}; c_6' = c_6u^{-6}; j' = j\]
Notice only $u$ affects $c_4,c_6,\Delta$.

\item For $Char(K) \neq 2$: $y = (y-a1*x-a3)/2$ 
\[ y^2 = 4*x^3 + b2*x^2 + 2*b4*x + b6\]
\[b2 = 4*a2 + a1^2; b4 = 2*a4 + a1*a3\] 
\[b6 = 4*a6+a3^2; b8 = b2*b6-b4^2\]
\[c4 = b2^2 - 24*b4; c6 = -1*b2^3 +36*b2*b4 - 216*b6\]
\[\Delta = 12^{-3}(c4^3-c6^2); j = c4^3/\Delta\]
\[\omega = \frac{dy}{2y + a_1x + a_3}\]
$\omega$ is non-vanishing with a simple pole at $\infty$.  

\item $Char(K) \neq 2,3$: $(\frac{x-3b_2}{36}, \frac{y}{108})$ 
\[W(A,B): y^2 = x^3 + (A= -27c_4)x + (B = -54c_6)\]
When $y^2 = f(x): \Delta_E = 16\Delta_{f(x)}$ and the 2-torsion points are of the form $(r,0)$ where $r$ is a root of $f(x)$.  

\item \textbf{EC/$\C$}: $E \cong \C/\Lambda; \Delta = g_2^3 - 27g_6^2$
\[(\wf')^2 = 4\wf^3 - (g_2 = 60G_4)\wf - (g_3 = 140G_6)\wf\]
\[z \lra [\wf(z):\wf'(z):1] \]
where \textbf{Weiestrass Function}: $\wf(z, \Lambda) = z^{-2} + \sum_{0 \neq w \in \Lambda} (z-w)^{-2} - w^{-2}$.

\item \textbf{Endomorphism and CM}: In $Char(K) = 0$: $End(E) \cong \Z$ or an order of the imaginary quadratic $\Q(w_2/w_1)$. In other characteristics, it is the later or an order of a quaternion algebra.

\item \textbf{Discriminant}: $\Delta = g_2^3 - 27g_3^3 =_1 (2\pi)^{12} q\prod_{n \geq 1}(1-q^n)^{24} =_2 (2\pi)^{12} \eta(\tau) =_3 (2\pi)^{12} \sum_{n \geq 1} \tau(n) q^n$ where 1 is the Jacobi product, 2 is the Dedekind function $\theta$, 3 is Ramanujan $\tau$. In the form $y^2 = f(x): \Delta_E = 16\Delta_{f(x)}$. 

\item \textbf{Ramanujan function $\tau$:} Multiplicative on gcd = 1, and relation* on primes $p$. \textbf{Deligne Thm:} $\forall n \geq 1: |\tau(n)| \leq \sigma_0(n) n^{11/2}$. \textbf{Lehmer's Conj.}: $\forall n \geq 1: \tau(n) \neq 0$.

\item \textbf{$j$-invariant}: $j: SL_2(\Z)\setminus \hh \ra \C/ \Gamma(1)\tau = j(\C/\Lambda_\tau) \in M_0(\Gamma(1))$ extends to give $X(1) \cong \PP^1(\C)$. Also $j: F \ra \C$ given by $j(\tau) = q^{-1} + \sum_{n \geq 0} c(n)q^n$ with $c:\N \ra \Z$.\\
Notice $f \in M_0(\Gamma(1)) \Ra f \in \C(j)$ and $f$ holomorphic in $\hh \Ra f \in \C[j]$. 

\item $j(E^\sigma) = j(E)^\sigma \Ra j(E)$ is an algebraic integer $\in \overline{\Q}$. \\ 
\[\forall j_0 \in \overline{K}: \exists E/K(j_0): j(E) = j_0\]

\item \textbf{Isogenies} (Special Algebra Homs): Surjective Homomorphisms with finite $n$-Kernel. All finite subgroups of $E$ are the kernel of an isogeny. 

In $\Q$: In Semistable (all good or multiplicative reduction), there are only $p = 2,3,5,7,13$ isogenies. For non-prime degrees $m \leq 10, 12,16,18,25; 14,15,21,27$, depending on $j$-invariant. See Cremona pg.99

The invariants are computed by finding a point in $E(\Q)[p]$ and then Velu's formulae or Laska's book formulae provide it. Much related to $p$-division polynomials, whose solutions are the $x$-coordinate of $p$-torsion points.

\item \textbf{Dual Isogeny}: $\varphi: E_1 \ra E_2$ and $\hat{\varphi}: E_2 \ra E_1$ with $\phi \circ \hat{\phi} = [\# Ker]_{E_2}$ and $\hat{\phi} \circ \phi = [\#Ker]_{E_1}$. 
\[\hat{\phi}: e \mapsto [e] \in Div(E_2) \mapsto \sum_{x \in \phi^{-1}(e)} e_x [x] \mapsto \sum e_x\cdot x\]

In the study of all $n$ degree isogenies: 
\[T(n),R_\lambda: Div(L) = \bigoplus_{\Lambda \in L}\Z \Lambda \ra Div(L)\]
\[T(n)\Lambda = \sum_{[\Lambda:\Lambda'] = n} \Lambda' \textrm{ and } R_\lambda \Lambda = \lambda \Lambda\]
Commute and $T$ satisfies the relation* on $p^e$'s. Correspond to 
\[\phi^*: \ACK(E_2) \ra \ACK(E_1)/ deg(\phi^*) = [\ACK(E_1):\ACK(E_2)]\]

\item \textbf{Holomorphic homomorphisms} (Algebra-Analytic Homs) $\lra \{\alpha \in \C: \alpha \Lambda_1 \subseteq \Lambda_2\}$.\\
Example: 
\[[\Lambda:\Lambda'] = n \Ra \]
\[\C/\Lambda' \ra C/\Lambda/ z \mapsto z\] 
\[\C/\Lambda \ra C/\Lambda'/ z \mapsto nz\] 

\item \textbf{Homotetic Lattices}: (Analytic Homs) $\alpha \Lambda_1 \cong \Lambda_2$. Isogenies correspond to $\alpha \Lambda_1 \subseteq \Lambda_2$. 

\item \textbf{Twists}: $\{E/K \cong_{\ACK} E'/K\}/\{E/K \cong_K E'/K\}$. For $Char(K) \neq 2,3$: 
\[Twist((E,0_E)/K) \cong K^*/K^{*n}; n = \left\{\begin{array}{cc} 2 & \La j \neq 0, 12^3 \\  4 & \La j = 12^3 \\ 6 & \La j = 0 \end{array}\right.\]
\[E = W(A,B) \Ra E_D = \left\{\begin{array}{cc} W(D^2A,D^3B) & \La j \neq 0, 12^3 \\  W(DA,0) & \La j = 12^3 \\ W(0,DB) & \La j = 0 \end{array}\right.\]



\item $Aut(E)$ automorphism with $0_E \mapsto 0_E$. $Isom(E) = Isom_{\ACK}(E)$ are general automorphisms without retriction. $Twists(E) = \frac{Isom_{\ACK}(E)}{Isom_K(E)}$. 
\[Twist(E/K) \lra H^1(G_{\ACK/K},Isom_{\ACK}(E))\]
where twist $\phi: C \ra E$ maps to $\xi(\sigma) = \phi^{\sigma}\phi^{-1}$. 

\item The $\#$ of \textbf{Automorphism Group}: is given by 
\[ \begin{array}{cccc} j\setminus Char(K) & \neq 2,3 & 3 & 2 \\
\neq 0,12^3 & 2 & 2 & 2 \\
=0 & 6 & 12 & 24 \\
= 12^3 & 4 & X & X \end{array}\]
Always divides $24$

\item For $Char(K) \neq 2,3$: $Aut(E) \cong \mu_n$ with $n = 2,4,6$ if $j \neq 0,12^3; = 12^3; = 0$. From $W(A,B)$ and the $Aut(E)$ can be given by $T(u,0,0,0)$ with unit $u$.

\item \textbf{Torsion Points}: $E_{tors} = \bigcup_m E[m] = E(\ACK)[m] = ker[m]$.\\
Assuming $char(K) = p$, then 
\[(m,p) = 1 \Ra E(\ACK)[m] \cong \Z/m\Z \oplus \Z/m\Z\]
\[m = p^e \Ra E(\ACK)[p^e] = 0 \textrm{ or } \Z/p^e\Z, \forall e\] 

\item \textbf{Tate Module}:
\[T_p(E) = \lim_{\overset{\la}{n \ra \infty}} E[p^n]\]
It is a $\Z_p$ module via $z(a_n)_n = ((z\,mod\,p^n)a_n)_n$.
\[l \neq Char(K) \Ra T_l(E) \cong \Z_l \times \Z_l\]
\[l = Char(K) \Ra T_l(E) \cong \{0\} \textrm{ or } \Z_l\]

\item The \textbf{$q_\beta$-Frobenius}: $\phi_{\beta,l}: T_l(\tilde{E}) \ra T_l(\tilde{E})$ thought as a matrix on a basis has characteristic polynomial the local $L$-series.

\item \textbf{$l$-adic representation of $G_{\overline{K}/K}$}: 
\[\rho_l: G_{\overline{K}/K} \ra Aut(T_l(E))\]

\item For $l \neq Char(K)$:
\[Hom(E_1,E_2)\otimes \Z_l \hra Hom(T_l(E_1), T_l(E_2))\]
Isomorphism for finite fields and number fields. 

\item Absolute Inertia group $I(\overline{K}/K)$, $l \neq Char(K) = p$. Let $V_l(E) = T_l(E) \otimes_{\Z_l} \Q_l$ be the $l$-adic Tate module of $E$, and $V_l(E)^{I(\overline{K}/K)}$ its fixed part by $I(\overline{K}/K)$. Then 
\[\textrm{Tame Part of }N: \epsilon(E/K) = \dim_{\Q_l}V_l/V_l^I = 2 - \dim_{\Q_l}V_l\]
\[\textrm{Wild Part of }N: \delta(E/K) =  \sum_{ i \geq 1} \frac{g_i}{g_0} dim_{\F_l}(E[l]/E[l]^{G_i})\]
\[f(E/K) = \epsilon(E/K) - \delta(E/K)\]
If $p \neq char(K)$, then
\[T_p(E) = \hat{H}_{et}^1(E \times_K K^{sep}, \Z_p)\]
where the hat means dual.

\item \textbf{Frobenious}: $Char(K) = q = p^n \Ra \phi: E \ra E^{(q)}/ (x,y) \mapsto (x^q,y^q)$. 
\[\Delta(E^{(q)}) = \Delta(E)^q; j(E^{(q)}) = j(E)^q\]

\item \textbf{Equivalent Categories}: 
\begin{itemize} 
\item Elliptic Curves$/\C$ with isogenies, 
\item EC$/\C$ with holomorphic homomorphisms. Let $ELL_\C$ be the set of $\C$-isomorphic classes of elliptic curves.  
\item $\C/\Lambda$ with homotetic maps $\alpha \Lambda_1 \subseteq \Lambda_2$. Let $L/\C$ be homotetic classes of lattices.
\end{itemize}
\[\begin{array}{ccc ccc cc}
\C & \lra & Y(1) = \Gamma(1)\setminus \hh & \lra & L/\C^* & \lra & ELL_\C \\
j(\tau) & \la & \tau & \ra & \{\Lambda_\tau\} & \ra & \{E_{\Lambda_\tau}\}
\end{array}\]

\item \textit{Singularities-Taylor Series}: $f(x,y) - f(x_0,y_0) = ((y-y_0)-\alpha(x-x_0))((y-y_0)-\beta(x-x_0)) - (x-x_0)^3$. With $\alpha,\beta$ the slopes of the tangents at the singularity.

\item \textbf{Reduction}: $E/K = W(a_i) \ra \tilde{E}(\F_q)$ or modulo $M$: 
\begin{itemize}
\item \textbf{Good reduction or stable}: $\tilde{E}$ nonsingular, $v(\Delta) = 0$. 
\item \textbf{Multiplicative reduction or semistable}: $\tilde{E}$ has \textbf{singular node} or $\alpha \neq \beta$, $v(\Delta) > 0$ or $\tilde{\Delta} = 0$ and $v(c_4) = 0$ or $\tilde{c_4} \neq 0$. Then $\tilde{E}_{ns}(\overline{k}) \cong \overline{k}^*$. If the tangents are $y = \alpha_i x + \beta_i$, then 
\[\tilde{E}_{ns} \raa{\cong} \overline{k}^*/ (x,y) \mapsto \frac{y - \alpha_1x-\beta_1}{y-\alpha_2 x - \beta_2}\]
\item \textbf{Split}: Slopes of node-tangents in $K$.
\item \textbf{Non-split}: Slopes of node-tangents not in $K$.
\item  \textbf{Additive reduction or unstable}: $\tilde{E}$ has a \textbf{singular cusp} or $\alpha = \beta$, $v(\Delta) > 0$ or $\tilde{\Delta} = 0$ and $v(c_4) > 0$ or $\tilde{c_4} = 0$. Then $\tilde{E}_{ns}(\overline{k}) \cong \overline{k}^+$. If the tangent is $y = \alpha x + \beta$ and $S$ is the cusp.
\[\tilde{E}_{ns} \raa{\cong} \overline{k}^*/ (x,y) \mapsto \frac{x - x(S)}{y - \alpha x - \beta}\]
Recall ns stands for the non-singular part, which is an abelian group. 
\end{itemize}

\item In \textbf{unramified} $K'/K$ the reduction type doesn't change. In a \textbf{finite} extension, good and multiplicative reduction stay the same. There is \textbf{an extension}, such that only good or multiplicative split reduction occurs.

\item \textbf{Serre-Tate}: $E/L$ with complex multiplication, then $E$ has potential good reduction at every prime of $L$. 

\item \textbf{Local Reduction}: $R \ra k = R/\pi R$: $E=W(a_i) \mapsto \tilde{E}=W(\tilde{a_i})$. On projective points $\tilde{P} = [\tilde{x},\tilde{y}, \tilde{z}] \in \tilde{E}(k)$.

\item \textbf{Hasse}: $E/\Q: |a_{q = p^n}| = |q+1 - \# \tilde{E}(\F_q)| \leq 2 \sqrt{q}$ 
\[f = \sum_{n \geq 1} a_nq^n \in S_2(\Gamma_0(E.conductor))\] 

\item \textbf{Local Minimal Equation}: minimal $v(\Delta)$ and  $a_i \in R$. Enough $v(\Delta) < 12$ or  $v(c_4) < 4$ or $v(c_6) < 6$. In Characteristics $\neq 2,3$ the converse holds. \\

\item \textbf{Minimal Discriminant} (Ideal): Minimal WE produces 
\[\D_{E/K} = \prod_{v \in M_K^0}\pp_v^{ord_v(\Delta_v)}\] 

\item \textbf{Weierstrass Class of $E/K$}: Reducing through valid transformations $T(u,r,s,t)$ to minimal WE:
\[\overline{\af}_{\Delta} = \overline{\prod_{v \in M_K^0} \pp_v^{-ord_v(u_v)}} \Ra \D_{E/K} = (\Delta)\af_{\Delta}^{12}\]
The class $\overline{\af_\Delta}$ of $\af_{\Delta}$ is well defined.

\item \textbf{Global minimal WE}: Simultaneously minimal for $M_K^0$: $\exists \equiv \overline{\af}_{E/K} = (1)$. On $K$ with class number $1$ (including $\Q$) all EC have GMWE. \textbf{Setzer}: On Quadratic Complex Fields $k = \Q[\sqrt{-m}]$, if $(h_k,6) = 1$, then all EC have GMWE and none EC have good reduction everywhere . 

\item Global Minimal Models on $\Q$ Normalized: $a1,a3 \in \{0,1\}$ and $a2 \in \{-1,0,1\}$. Not Unique and related so $u = \pm 1$ and $r,s,t \in \Z$. 

\item \textbf{Quadratic Forms}: $d: E \ra \R$ even and the pairing $E \times E \ra \R/(P,Q) = d(P+Q) - d(P) - d(Q)$ is bilinear.

\item \textbf{Heights on $\PP^N$}:\\
\[H_\Q(P/\Q) = \max(|x_i^\Z|); h(P/\Q) = log(\max|x^\Z_P|)\]
\[H_K(P/K) = \prod_{v \in M_K} \max(|x_i|_v)^{[K_v:\Q_v]} \geq 1\]
\textbf{Absolute Height/ Absolute Logarithmic Height}: 
\[H_{\overline{\Q}}(P) = H_K(P/K)^{1/[K:\Q]}; h_{\overline{\Q}} = \log H_{\overline{\Q}}\]
\[H \circ Gal(\overline{\Q},\Q) = H\]  
For $F = (f_i):\PP^N(\overline{\Q}) \ra \PP^M(\overline{\Q})$/ $deg(F) = deg(f_i) = d$ with no common zero but $(0,\ldots,0)$. Then $\exists C_1,C_2$ depending only on $F$ with
\[C_1H(P)^{d} \leq H(F(P)) \leq C_2 H(P)^d\]

\item \textbf{Heights on EC}: 
\[f \in \ACK(E/K) \Ra \exists f: E \ra^{surj} \PP^1(\ACK)\] 
\textbf{Relative Height}: $h_f(P) = h(f(P))$.

\item \textbf{Canonical Neron-Tate height}: 
\[\hat{h}: E(\overline{K}) \ra \R_{\geq 0}/\hat{h}^{-1}(0) = E(\overline{K})_{tor}\]
\[\hat{h}(P) = \frac{1}{deg(f)} \lim_{N \ra \infty} 4^{-N} h_f([2^N]P)\] 
where $f \in K(E)$ is any nonconstant even function. Satisfies \textbf{Parallelogram Law}, $\hat{h}([m]P) = m^2 \hat{h}(P)$. $\hat{h}$ extends to positive definite quadratic form on $E(K) \otimes \R$, with $E(K)_{tors}$ as a lattice in $E(K) \otimes \R$. 

\item \textbf{Naive Weil height}: $P = (a/c^2,b/c^3) \Ra h(P) = \log\max\{|a|,c^2\}$. Related by $\hat{h}(P) = \lim_{n \ra \infty} 4^{-n}h(2^nP)$. 

\item \textbf{Algorithm-Canonical Height EC$/\Q$}: 
\textbf{Good Reduction at $p$ or reduction to a non-singular point}: For a minimal equation at $p$: If $ord_p(3x^2 + 2a_2x + a_4 - a_1y) \leq 0$ and $ord_p(2y+a_1x + a_3) \leq 0$, then $\hat{h}_p(P) = \max\{0, -ord_p(x)\}\log(p)$. \\
\textbf{Multiplicative}: If $ord_p(c_4) = 0$, set $N = ord_p(\Delta)$ and $M = \min\{\psi_2(P), N/2\}$, then $\hat{h}(P) = \frac{M(M-n)}{N}\log(p)$. \\
\textbf{Additive, Type IV or IV$^*$}: If $ord_p(\psi_3(P)) \geq 3ord_p(\psi_2(P))$, then $\hat{h}_p(P) = -2/3ord_p(\psi_2(P))\log(p)$. 
\textbf{Additive, Type III or III$^*$, I$_m^*$}: $\hat{h}_p(P) - -1/4 ord_p(\psi_3(P))\log(p)$. \\

\textbf{Global Height}: If $P = (a/c^2,b/c^3)$, then 
\[\hat{h}(P) = \hat{h}_\infty(P) + 2\log(c) + \sum_{p|\Delta, p \not|c} \hat{h}_p(P)\]
See Cremona AfMF pg73-74 for computing $\hat{h}_\infty(P)$.

\item \textbf{Elliptic Regulator} or volume of the fundamental domain $E(K)/E_{tors}(K)$
\[R_{E/K} = det(\langle P_i, P_j \rangle) > 0\] 

\item \textbf{Conjecture on Minimal Height} $> 0$: $\exists C \equiv C([K:\Q])>0: \forall P \in E(K)-E_{tors}$: 
\[\hat{h}(P) > C \max\{h(j_E), \log \, N_{K/\Q}\DD_{E/K}, 1\}\]
This has been proved with (1) $C$ depending also on the number of places $v \in M_K^0/ ord_v(j_E) < 0$ (number of primes dividing the denominator of $j_E$). (2) Assuming ABC for $K$, $C$ depending also on the exponent and constant appearing in the ABC conjecture. 

\item \textbf{Frey's Curve}: $a+b = c$ and $E:y^2 = x(x+a)(x-b)$.

\item \textbf{Conductor}: Invariant under Isogenies 
\[N_E = \prod_{p\textrm{ prime}} p^{f_p(E)}\]
\[f_{p> 3}(E) = 0_{good},1_{mult},2_{add}\]
and $f_2(E) \leq 5, f_3(E) \leq 3$.\\


\item \textbf{Zeta Function}: 
\[Z(E/\F_q;T) = \exp\left(\sum_{n \geq 1} \#E(\F_{q^n})\frac{T^n}{n}\right)\]
Notice not defined like $\sum \#E(\F_{q^n})T^n$. (Generalize for Projective Varieties)

\item \textbf{Weil's Conjectures}:\\
\textit{Rationality}: $Z(V/\F_q; T) \in \Q(T)$.\\
\textit{Functional Equation}: $\exists \epsilon \in \Z$ (the Euler characteristic of $V$): $Z(V/\F_q; 1/q^NT) = \pm q^{N\epsilon/2} T^\epsilon Z(V/\F_q; T)$. \\
\textit{Riemann Hypothesis}: $Z(V/\F_q;T) = \frac{P_1\ldots P_{2N-1}}{P_0\ldots P_{2N}}$ with $P_i \in \Z[T]$, $P_0 = 1-T$, $P_{2N} = 1-q^NT$, the $b_i$ roots of $P_i$ have absolute value $q^{1/2}$. The $b_i$'s are called Betti Numbers of $V$.

\item 
\[Z(E/\F_q;T) = \frac{1-aT+qT^2}{(1-T)(1-qT)}\]
The roots of numerator have absolute value $\sqrt{q}$, $a = q+1 - \#E(\F_q)$ is the trace of the Frobenius, $q$ is the determinant of the Frobenious. 
\textbf{Functional Equation}: $\epsilon = 0 \Ra$
\[ Z(E/\F_q; 1/qT) = Z(E/\F_q; T)\]


\item \textbf{Mordell-Weil}: $E/K \cong E_{tor} \times \Z^r$ is a finitely generated group. \\

\textbf{Siegel}: Only finitely many integral points.\\ 

\textbf{Faltings}: $C_{g > 1}(K) < \infty$.\\ 

\item \textbf{Mazur}: $E(\Q)_{tor} \in \Z/n\Z$ for $1 \leq n \leq 10$  or $12$ if $\Delta < 0$; $\Z/2\Z \oplus \Z/2n\Z$ for $1 \leq n \leq 4$ if $\Delta > 0$.\\
Notice we can have torsion free curves but not everywhere good reduction.\\

\item \textbf{Tate Normal Form}: These families guarantee to have an $n$ torsion point but that doesn't mean $n$ is the largest such. 
\[Y^2 + (1-c)XY - bY = X^3 - bX^2\]
$n = 4 \Ra b = \alpha, c = 0$.\\
$n = 5 \Ra b = \alpha, c = \alpha$.\\
$n = 6 \Ra b = \alpha + \alpha^2, c = \alpha$.\\
$n = 7 \Ra b = \alpha^3 - \alpha^2, c = \alpha^2 - \alpha$.\\
$n = 8 \Ra b = (2\alpha-1)(\alpha-1), c = b/\alpha$.\\
$n = 9 \Ra b = c(\alpha^2 - \alpha + 1), c = \alpha^2(\alpha-1)$.\\
$n = 10 \Ra b = c\alpha^2/[\alpha - (\alpha-1)^2], c = (2\alpha^3 - 3\alpha^2 + \alpha)/[\alpha - (\alpha-1)^2]$.\\
$n = 12 \Ra b = c(2\alpha - 2\alpha^2 - 1)/(\alpha-1), c = (3\alpha^2 - 3\alpha + 1)(\alpha - 2\alpha^2)/(\alpha-1)^3$.\\

\textbf{Lutz Nagell}: $E/\Q = W(A,B)$ and $0 \neq P \in E(\Q)_{tor} \Ra P \in \Z^2$. And either $y(P) = 0 \Ra [2]P = 0$ or $y(P) \neq 0 \Ra y(P)^2|4A^3 + 27B^2$. More general for any number field $K$: $x(P),x([2]P) \in R \Ra y(P)^2|4A^3 + 27B^2$. \\

\textbf{Merel}: $\forall d \geq 1: \exists N(d): \forall [K:\Q] \leq d: |E_{tors}(K)| \leq N(d)$.\\

\textbf{Manin}: $\forall p: |E_{tors}(K)[p]| \leq N(K,p)$.

\item \textbf{Conjecture on Ranks}: $E/\Q: \forall \epsilon > 0: rank(E) = r: \exists$ basis $P_1,\ldots,P_r$ of $E(\Q)_{free}$ with 
\[\max_{1 \leq i \leq r} \hat{h}(P_i) \leq C_\epsilon^{r^2}|D_{E/\Q}|^{\frac{1}{12}+\epsilon}\]

\item \textbf{Classical Result}: All EC with given Torsion subgroup lie in a one-parameter family, i.e.: $\exists d \in K: a_i \equiv a_i(d)$. More general $E/K$ with a point $P \in E(K)$ of order $m \geq 4$ are parametrized by the $K$-rational points of another curve called modular curve. 


\item \textbf{Main Theorem on Torsion Points}: $E = W(a_i \in \OO_K): ord(P) = m \geq 2$: $m \neq p^n \Ra x(P),y(P) \in R$ and $m = p^n \Ra \forall v \in M_K^0:$
\[r_v = \left\lfloor \frac{ord_v(p)}{p^n - p^{n-1}}\right\rfloor \Ra ord_v(x(P),y(P)) \geq -(2,3)r_v\]

\item \textbf{2-Descent}: Can assume $E[2] \subseteq E(K)$ or work in the splitting field of the 2-divisor polynomial $\psi_2$. Write $E: y^2 = \prod(x-e_i)$ with $e_i \in K$. Let $S = M_K^\infty \cup M_K^0(\pp|2) \cup M_K^0(\pp$ bad reduction). Then $K(S,2) = \{b \in K^*/K^{*2}: \forall v \notin S: 2|ord_v(b)\}$. Since we are moding squares, then $K(S,2)$ are elements whose factors are some primes in $S$ without repetition.
\[E(K)/2E(K) \hra K(S,2)^2; P = (x,y) \mapsto \]
\[(x-e_1,x-e_2); (e_1-e_3/e_1-e_2,e_1-e_2); (e_2-e_1, e_2-e_3/e_2-e_1); (1,1)\]
where the last three are images of $x =e_1,x=e_2, P = \infty$. We want to identify the subgroup of $K(S,2)^2$ that is the image. $(b_1,b_2) = Im(P) \Lra b_1z_1^2 - b_2z_2^2 = e_2 -e_1$ and $b_1z_1^2 - b_1b_2z_3^2 = e_3 - e_1$ has a solution in $K^{*2} \times K$ and $P = (b_1z_1^2 + e_1, b_1b_2z_1z_2z_3)$.  

\item \textbf{Homogeneous Spaces}: Twist $C$ with simple-transitive action of $E$ on $C$. For lower letter elements of $C$ and upper letter elements of $E$: $p + O_E = p$. Notice that $E$ is a homogeneous space of itself. 

\item \textbf{Equivalence of Homogeneous Spaces}: $C/K \cong C'/K$ iff $\cong_K$ and compatible with $E$-action, i.e.: $p \lra p' \Ra p + P \lra p' + P$. Trivial or in the class of $E$ iff $C(K)$ is not empty. 

\item \textbf{Weil Chatelet Group}: $WC(E/K)$ is the collection of equivalent classes of homogeneous spaces. 

\[WC(E/K) \lra H^1(G_{\ACK/K}, E)\]
where the class of the homogeneous space $C$ is sent to $(\sigma \mapsto p_0^\sigma - p_0)$ where $p_0 \in C$. 

\item \textbf{Selmer and Shafarevic}: Isogeny $E \raa{\phi} E'$ and 
\[t: G(\ACK/K) \times \{\frac{x}{m} \in E(\ACK)\} \ra E[m]/(\sigma,y_x = x/m) \mapsto y^\sigma - y\]
SES $E[\alpha], E(\ACK), E'(\ACK)$ has LES reducing to 
\[0 \ra E'(k)/\phi E(k) \ra H^1(G_K, E[\phi]) \ra H^1(G_K,E)[\phi] \ra 0\]
\[0 \ra \prod_{M_K} E'(K_v)/\phi E(K_v) \raa{\delta} \prod_{M_K} H^1(G_{K_v}, E[\phi]) \ra \prod_{M_K} WC(E/K_v)[\phi] \ra 0\]

where $H^1(G_K,E)[\phi] = Ker[H^1(G_K,E(\ACK)) \ra H^1(G_K,E'(\ACK))] \cong WC(E/K)$. Localizing, $G_{K_v} \subseteq G_K$ and $E(\ACK) \subseteq E(\ACK_v)$ makes a ladder restricting global-to-local sequence. Our goal is to compute $H^1(G_K,E[\phi])$. We need to extend each $v \in M_K$ to $\ACK$, but the results are independent of this extension. 

\item \textbf{Selmer Group}: \textbf{Effectively Computable} and finite. \[Sel^{(\phi)}(E/K) :=\bigcap_{M_K} Ker[H^1(G_K,E[\phi]) \ra \prod_{M_K} WC(E/K_v)\]

\item \textbf{Shafarevic Group} of $E/K$ is the subgroup of $WC(E/K)$
\[Sha(E/K) = Ker(WC(E/K) \ra \prod_{M_K}WC(E/K_V))\]

SES: where the selmer group is finite. 
\[0 \ra E'(K)/\phi E(K) \ra S^{(\phi)}(E/K) \ra Sha(E/K)[\phi] \ra 0\]

\item \textbf{Unramified Cohomology class} at $v \in M_k$: $\phi \in H^1(G_k,M)/ \phi|_{I_v} = 1$. Then $Sel^{(\alpha}) \subseteq H_S^1(G_k,Ker(\alpha))$, i.e. the cohomology classes unramified outside $S$ containing all archimedean, all bad reduction of $E$ and $E'$, and all places dividing $deg(\phi)$.  

 
\item Most important when dealing with multiplications by $n$:
\[0 \ra E(\Q)/nE(\Q) \ra S_n \ra Sha[n] \ra 0\]
Since $S_n$ is finite, it gives both a boundary to rank and $Sha[n]$ is finite.
\item The \textbf{Tate Shafarevic conjectured}: $Sha$ is finite. Proven for CM curves over $\Q$ with analytic rank $0$ or $1$ by Rubin and Kolyvagin in 1987.

\item $E \times Aut(E) \cong Isom(E)$ where $(P,\phi) \mapsto T_P \circ \phi$ where $T_P$ is translation by $P$ (adding $P$). The group operation on the left is given by $(P,a)(Q,b) = (P+aQ, ab)$.  
\[Twist((E,O_E)/K) = H^1(G_K,Aut(E)) \subseteq H^1(G_K,Isom(E)) = Twist(E/K)\]

\item \textbf{Notation}: $C$ being a twist of $(E,O_E)$ means $C \in H^1(G_{\ACK/K},Aut(E))$

\subsection{Complex Multiplication}
\item $K$ imaginary quadratic, then the set of orders of $K$ is given by $\Z + f_{\geq 1}\OO_K$.

\item Imaginary $[K = End(E) \otimes \Q:\Q] = 2, R = \OO_K \cong End(E)$; Notice $I \subseteq O_K \lra \Lambda' \subseteq \Lambda$.
\item  $\forall \alpha \in R: deg[\alpha] = |N_\Q^K\alpha|$. 

\item Normalized EC: $[\cdot]: R \raa{\cong} End(E)$ with $\forall \omega \in \Omega_E: [\alpha]^* \omega = \alpha \omega$. 
\[\begin{array}{cccc}
\C/\Lambda: & z & \raa{\alpha} & \alpha z \\
& \da && \da \\
 E_\Lambda: & (\wf(z), \wf'(z))  & \raa{[\alpha]} & (\wf(\alpha z), \wf'(\alpha z))\end{array}\]
This normalized pair respect isogenies: $\phi: E_1 \ra E_2 \Ra \phi \circ [\cdot]_1 = [\cdot]_2 \circ \phi$.\\

\item $ELL(O_K) = \frac{\{E/End(E) \cong R\}}{\C-ismorphism}$ has injective transitive action by $Cl(\OO_K)$, so $\#ELL(O_K) = Cl(O_K)$.
\[Cl(R) \times ELL(R) \ra ELL(R)\]
\[\af \times \{E_\Lambda\} \mapsto \{E_{\af \Lambda}\}\]

\item Homomorphism $F: Gal(\overline{K}/K) \ra Cl(O_K)$/$F(\sigma)$ is the unique element $\overline{\af}$ sending $E \mapsto E^\sigma$. Defined on $Gal(\overline{\Q}/\Q)$ but independent of $E$ only on $Gal(\overline{K}/K)$.\\
$j(\Lambda)^\sigma  j(F(\sigma)^{-1}\Lambda)$. \\
\[ (\overline{\af} \times E)^\sigma = \overline{\af}^\sigma \times E^\sigma \]

\item \textbf{Conjugates of $j(\Lambda)$} for $G(\overline{K}/K)$ are $j(E_i \lra \overline{\af}_i\Lambda)$ where $ELL(\OO_K) = \{E_i\}$.

\item $E/L$ with CM by $K \Ra End(E)/LK$.\\
$E_1,E_2/L \Ra \exists [L':L] < \infty: Hom(E_1,E_2)/L'$.

\item \textbf{Hasse}: if $E$ has good reduction at all primes of $K(j(\Lambda))$ lying over $\pp$: $$j(\Lambda)^{Frob(\pp)} = j(\Lambda \cdot \pp^{-1})$$

\item \textbf{Weber Function}: 
\[\phi: E = W(A,B) \ra \frac{E}{Aut(E)} \cong \PP^1\] 
\[\phi_E(P) = \left\{\begin{array}{l} x \La j(E) \neq 0, 12^3 \textrm{ or } AB \neq 0 \\  x^2 \La j(E) = 12^3 \textrm{ or } B = 0 \\ x^3 \La j(E) = 0 \textrm{ or } A = 0 \end{array}\right.\]
\[f: \C/\Lambda \ra E/ h(f(z)) = \left\{\begin{array}{l} (g_2g_3/\Delta)\wf(z) \La j(E) \neq 0, 12^3\\  (g_2^2/\Delta)\wf(z)^2 \La j(E) = 12^3 \\ (g_3/\Delta)\wf(z)^3 \La j(E) = 0 \end{array}\right.\]


\item For $K/\Q$ imaginary quadratic:\\
Hilbert CF = Maximal Abelian Unramified Extension: $K(j(E))$.\\
$E$ with CM by $\OO_K$, $h: E \ra \PP^1$ a Weber Function, $c$ integral ideal in $\OO_K$: The ray class field $K$ modulo $c$ is $K(j(E), h(E[c]))$. \\ 
Maximal Abelian Extension: $K^{ab} = K(j(E); h(E_{tor}))$.\\
$h_K = 1 \Ra K = H_K = K(j(E)) \Ra K^{ab} = K(h(E_{tor})) = K(E_{tor})$. 

\item \textbf{Important Extensions}: \[[K(j(\Lambda)):K] = [\Q(j(\Lambda)):\Q] = h_K\]

\[K \subseteq H(K) = K(j(E)) \subseteq \]
\[K^{ab} = K(E_{x-tors}) = K(j(E),\phi_E(E_{tors}))\]\[\subseteq_{ab} K(E_{tors}) \subseteq_{ab} K(j(E),E_{tors})\]
$\frac{K(E_{tors})}{K}, \frac{K(j(E),E_{tors})}{K^{ab}}, \frac{K(j(E),E_{tors})}{K(j(E))}$ abelian. Moreover, Actions of $G_{\overline{K}/K}$ and $End_K(E)$ commute on the Tate $T_l(E)$.

\item The Function fields over $\C$ are well presented in Diamond Fig 7.2 pg 285.:
\[\C(j, E_j[N]) \supseteq \C(X(N)) = \C(j,f_{1,0},f_{0,1}) \supseteq \C(X_1(N)) = \C(j,f_{1,0})\]
\[\supseteq C(X_0(N)) = C(j,j_N) = \C(j,f_{0,1}) \subseteq \C(X(1)) = C(j)\]
The Galois Groups are:
\[Gal \, \C(j,E_j[N]))/C(X_1(N)) = \{\pm (1,b,0,1)\}\]
\[Gal \, \C(j,E_j[N]))/C(X_0(N)) = \{\pm (a,b,0,d)\}\]
\[Gal \, \C(j,E_j[N]))/C(X(1)) = SL_2(\Z/N\Z)\]

\textbf{The Function fields over $\Q$} are presented in Diamond Fig 7.5 pg 289.


\item For Torsion $M \in R_K-mod$:  
\[Sum: \bigoplus_\pp M[\pp^\infty] \raa{\cong} M \Ra K/\af \cong \bigoplus_\pp (K_\pp/\af_\pp \cong K/\af[\pp^\infty])\]


\item Multiplication $A_K^* \times I_K \ra I_K$ given by $(x,\af) \mapsto x\af = (x)\af$ passing to ideals first and $K/x\af \cong \bigoplus_\pp K_\pp/x_\pp\af_\pp$. Hence, 
\[\forall x \in A_K^*: mult_x: K/\af \ra K/\af\]
\[K/\af \cong \bigoplus_\pp K_\pp/\af_\pp \raa{\bigoplus \times x_\pp} \bigoplus_\pp K_\pp/\af_\pp \ra K/\af\]
 
\item \textbf{Main Theorem of CM}: $K/\Q$ quadratic imaginary, $E/\C \cong \C/\af$ with $End(E) \cong R_K$ and $\af \in I_K$, $\sigma \in Aut(\C)$, $s \in A_K^*/ [s,K] = \sigma|_{K^{ab}}$. Then 
\[\begin{array}{rcl}
K/\af & \raa{s^{-1}} & K/s^{-1}\af \\
\da f & & \da f' \\
E(\C) & \raa{\sigma} & E^{\sigma}(\C)
\end{array}\]
Remains true if $End(E) \subsetneq R_K$ but needs to redefine multiplication by $x \in A_K^*$. 

\item \textbf{Associated Grossencharacter} CM-EC:\\
Grossencharacter to number field: continuous $\psi: A_L^* \ra \C^*$ with $L^* \subseteq Ker(\psi)$. \\
$E/L$ with CM over $K \subseteq L$, $x \in A_L^*, N_K^Lx = s \in A_K^*$. Then $\exists \alpha_{E/L}(x) \in K^*$: $\alpha R_K = (s)$. Then $\forall f: \C/\af \raa{\cong} E(\C)$. 
\[\begin{array}{rcl}
K/\af & \raa{\alpha s^{-1}} & K/\af \\
\da f && \da f \\
E(L^{ab}) & \raa{[x,L]} & E(L^{ab})
\end{array}\]
Notice $\alpha: A_L^* \ra K^* \subseteq \C^*$ sends a $L$-adele to a generator of the ideal obtain by taking norms to $K$-adeles and then to $K$-ideals. Yet $\alpha(L^*) \neq 1$. Indeed, $\beta \in L^*\mapsto N_K^L \beta$.\\

The actual \textbf{Grossencharacter} of $L$ is $\psi_{E/L}: A_L^* \ra \C^*$ given by $\psi_{E/L}(x) = \alpha_{E/L}(x) N_K^L(x^{-1})_\infty$.\\

\item The Grossencharacter $\psi_{E/L}$ is \textbf{unframified} at prime $\beta \in I_L$ iff $\psi(R_\beta^*) = 1$ and iff $E$ has good reduction at $\beta$. If $\psi$ is unramified at $\beta$, we define $\psi(\beta) = \psi(\ldots,1,\pi_\beta,1,\ldots)$, and  if $\psi$ is ramified at $\beta$, then $\psi(\beta) = 0$ .

\item $E/L$ and $\F_\beta = O_L/\beta$ and $q_\beta = N_\Q^L \beta = \# \F_\beta$. If $E$ has good reduction at $\beta$, we have $a_\beta = q_\beta + 1 - \#\tilde{E}(\F_\beta)$. And the factor $L_\beta(E/L,T) = 1 - a_\beta T + q_\beta T^2$ is called the local $L$-series. Also $L_\beta(E/L,T) = 1-T,1+T,1$ if $E$ has Split Multiplicative, Non-Split Multiplicative, Additive reduction resp.\\


\item The Global $L$-series $E/L$ is $L(E/L,s) = \prod_\beta L_\beta(E/L,q_\beta^{-s})^{-1}$ analytic for $Re(s) > 3/2$. Conjecture: Analytic continuation to $\C$ and satisfies functional equation relating values $s$ and $2-s$. Conjecture is true for EC with complex multiplication: If $\mathfrak{c}$ is the conductor of $\psi_{E/L}$ and $\mathfrak{c}'$ the conductor of $\psi_{E/LK}$.
\[\Lambda(E/L,s) = (N_\Q^{L}(\DD_{L/\Q}\mathfrak{c}_\psi)^s((2\pi)^{-s} \Gamma(s))^{[L:\Q]} L(E/L,s)\]
\[\Lambda(E/L,s) = (N_\Q^{LK}(\DD_{LK/\Q}\mathfrak{c}'_\psi)^{s/2}((2\pi)^{-s} \Gamma(s))^{[L:\Q]} L(E/L,s)\]
\[\Ra \Lambda(E/L,s) = \omega_{E/L} \cdot \Lambda(E/L,2-s)\]
where $\omega_{E/L} = \pm 1$ is the sign of the functional equation. 


\item Hecke $L$-series to Grossencharacter $\psi: A_L^* \ra \C^*$ is $L(s,\psi) = \prod_{\beta \in O_L} (1- \psi(\beta)q_\beta^{-s})^{-1}$. \\

\textbf{Hecke Thm}: $L(s,\psi)$ has analytic continuation to all $\C$ and there is a functional equation relating $L(s,\psi)$ and $L(N-s,\overline{\psi})$ for some real number $N = N(\psi)$.\\

Recall we defined $\phi_\beta$ associated to the $q_\beta$-Frobenius and $\psi$ on primes $\beta$:
\[\begin{array}{rcl}
E & \raa{[\psi_{E/L}(\beta)]} & E\\
\da & & \da \\
\tilde{E} & \raa{\phi_\beta} & \tilde{E}
\end{array}\]
Furthermore, $q_\beta = N_\Q^L\beta = N_\Q^K(\psi_{E/L}(\beta))$.\\
$\#\tilde{E}(\F_\beta) = N_\Q^L\beta + 1- \psi_{E/L}(\beta) - \overline{\psi_{E/L}(\beta)}$.\\
$a_\beta = \psi_{E/L}(\beta) + \overline{\psi_{E/L}(\beta)}$.\\

Deurings Thm: $E/L$ with CM by $O_K$ (1) If $K \subseteq L$, then $L(E/L,s) = L(s,\psi_{E/L})L(s,overline{\psi_{E/L}})$. (2) If $K \not \subseteq L$ and $L' = LK$, $L(E/L,s) = L(s, \psi_{E/L'})$. 

\subsection{Divisors}
\item \textbf{Weil Divisors} are defined on Noetherian, Integral = Irreducible and Reduced, Separated, Regular in codim 1 = $\forall x: \mm_x$ is principal. $Div(X) = \bigoplus$ prime divisors or closed integral codimension-1. Example: $Cl(X \times \AA^1) \cong Cl(X)$. Main results $Cl(\PP^n) = \Z\cdot Z(x_0) \cong \Z$.

\item \textbf{Cartier Divisors}: Consider the sheaf(ification of) $\mathcal{K}: U \mapsto Frac(\Gamma(U,\OO_X))$ (zero divisors are not invertible, so not a field). Then $CaDiv(X) = \mathcal{K}^*/\OO_X^*$.  An element $(U_i,f_i \in \Gamma(U_i,\mathcal{K}^*/ f_if_j^{-1} \in \Gamma(U_i \cap U_j,\OO_X^*)$. 

\item \textbf{Picard Group}: \textbf{Invertible Sheafs} are a group under $\otimes$ with inverse $\mathcal{L}^{-1} = \mathcal{L}^\sim = Hom(\mathcal{L},\OO_X)$ called $Pic(X)$.   


\item Integral, Separated, Noetherian, $\forall x: \OO_x$ UFD means:
\[Cl(X) \equiv CaCl(X) \equiv_{always} Pic(X)\]

From Cartier Divisors $D = (U_i,f_i)$ to Weil Divisors: Consider Every prime divisor $Y$ and then $D \mapsto \sum_Y v_Y(f_i)Y$ (independent of what $i$ is selected).\\

From Weil Divisor $D = \sum n_i Y_i$ to Cartier Divisors: Take the restriction $D_x \in Div(Spec(\OO_x)) = PDiv(Spec(\OO_x) \Ra D_x = (f_x)$ which end up being local in $U_x$. Then cover $X = \bigcup U_x$ and the cartier divisor is $(U_x,f_x)$. \\  

From Cartier Divisors $D = (U_i,f_i)$ to Invertible Sheafs: $\mathcal{L}(D) \subseteq \mathcal{K}^*: (U_i,\langle f_i^{-1}\rangle = f_i^{-1}\OO_X)$. Sum corresponds to tensor and difference to inverse.\\

From Invertible Sheafs to Cartier Divisors: $\mathcal{L} \subseteq \mathcal{K} = \mathcal{L} \otimes \mathcal{K}$.\\

\item $Cl(\PP^n) = \Z \cdot Z(x_0)$. Then $Pic(\PP^n) = \{\OO_X(l)\}_l$. 

\item On Non-singular Elliptic Curves all agree. $Cl(X) \cong CaCl(X) \cong Pic(X)$ but we think of them as Weil Divisors $\bigoplus_{e \in E} \Z\cdot e$. 

\textbf{Principal}: $(f) = \sum ord_e(f)[e]$.\\
\textbf{Linear Equivalence}: $D \sim D'/ D-D'$ principal.\\

\item $L(D \in Div(X)) = \{f \in k(X)^*/ D + (f) \geq 0\} \cup \{0\}$ with dimension $l(D)$.
\item \textbf{Cannonical Divisor}:$K = (f)/f$ is a global meromorphic $1$-form. For EC: $K = (\omega) = 0$. 
\item \textbf{Riemann Roch Theorem}:
\[l(D) + l(K-D) = deg(D) - g + 1\]

\item \textbf{Complete Linear Systems} in bijection with $L(D)$'s:
\[|D| = \{D' = D + (f) \geq 0\} \lra \{f \in k(X)^*/ D + (f) \geq 0\} = L(D)\]
\[L(D) = \bigoplus^{l(D)}k \cdot f_i\]
\[\PP(L(D)) = L(D)^*/k^* \cong \PP^{l(D) - 1}\]
\[\PP(L(D))  \ra |D|/(a_0,\ldots,a_{l(D)-1}) \sim f = \sum a_if_i \mapsto D + (f)\]
\textbf{Linear Systems} $\subseteq |D|$ parametrized by $\PP(V) \subseteq \PP(L(D))$ with $V \subseteq L(D) \subseteq k(X)^*$.\\
\textbf{Rational Map associated to a Linear System}: Choose basis $f_0, \ldots, f_n$ of $V$
\[\phi_L: X \ra \PP^n/x \mapsto (f_0(x),\ldots,f_n(x))\]

\item For a Form $F \notin I_X$ of degree $d$:
\[L_X(d):= \{(F)_X/ F \in (k[x_i]/I_X)^*_d  \}\]

\item \textbf{Base Points of a Linear System}: For $L \subseteq |D| \subseteq Div(X)$, they are the intersection of supports of divisors in $L$. \textbf{Base point Free} means this intersection is empty. \textbf{base point free divisor} $D$/ $|D|$ is base point free.

\item \textbf{Fixed Component} of $L$ is $\max\{D_0/ L - D_0 \geq 0\}$. If $D_0 = 0$, we say $L$ has no fixed component. 

\item \textbf{THM}: bijection between linear systems $L$ of dimension $n$ without fixed components, and morphisms $X \ra \PP^n$ whose image is not contained in a hyperplane up to projective automorphism. 


\subsection{Neron Models}
\item \textbf{Group Varieties} $G$: Non-singular, connected $\Ra$ irreducible, and $1_G \in G^0 \triangleleft G$.
\item \textbf{All connected GV are linear groups}: $\subseteq GL_n$ and $\forall n \geq 2:$ noncommutative.

\item \textbf{All $1$-dim GV} are EC, $\mathbb{G}_a \cong \AA^1$, $\mathbb{G}_m \cong \AA^{1*}$. 

\item \textbf{Rigidity Lemma}: $\phi: G_1 \times G_2 \ra G_3/ \#\phi(e_1 \times G_2) = \#\phi(G_1 \times e_2) = 1 \Ra \#Im(\phi) = 1$.

\item \textbf{$S$-schemes}: $X \ra S$.\\
Fibres $S \ra X$.\\
For $X \ra T$: \textbf{$T$-valued points}: $X(T) = Hom_S(T,X)$. Agrees with $K$-valued points definition for $X = Z_K(f_i)$:
\[X(Spec(K)) = Hom_K(Spec(K),X)\]\[\cong Hom(K[x_i]/(f_j), K) = Z_K(f_i) = X(K)\] 

\item \textbf{Fibre Products}: $X\times_S Y, p_X = p_1, p_Y = p_2$ with universal property $Z \hra X \times_S Y$.\\
\textbf{Diagonal}: $\delta_X: X \ra X \times_S X$.\\
\textbf{Graph}: $\delta_{\phi}: X \ra X \times_S Y$.\\

\item \textbf{Special Examples of Fibre Products}:\\
(1) $R$ DVR: $X_\mm = X \times_R \mm = X$ reduced mod $\mm$, which is scheme over $R/\mm$. \\
(2) $R$ DVR: $\eta = (0)$ !generic point for $Spec(R)$. Generic Fibre $X_\eta$ and Special or Closed Fibre $X_\mm$. 

\item \textbf{Regular Scheme}: Nonsingular, i.e.: $\forall P: dim(\mm_{X,P}/\mm^2_{X,P}) = dim_{Krull}(\OO_{X,P})$.

\item \textbf{Proper Morphism}: Complete and separated fibres. Valuative criterion for finite type morphisms between Noetherina schemes: $\forall R$ (DVR) with $Frac(R) = K$, $Spec(K),X,Spec(R),S: \exists! Spec(R) \ra X$. All Closed subschemes in $\PP^n_{S:Noeth}$ are proper. 

\item \textbf{Smooth Morphism}: nonsingular fibres or domain is a family of regular schemes. For $R$ (DVR) and $X$ integral (reduced and irreducible) of finite type: $X_\eta \neq \emptyset$: $X$ smooth iff $X_\eta(\ACK),X_\mm(\ACK)$ are nonsingular. 

\item \textbf{Group Scheme}: Scheme whose fibres form an algebraic family of groups (only elements in the same fibre can be operated) $\pi: G \ra S$ and $\sigma_0: S \ra G$ (all send to identity) together with $\mu: G \times_S G \ra G$ (notice multiplication is defined on fibre not on the whole product $G \times G$), $i: G \ra G$. 

\item Examples:\\
$\mathbb{G}_a = Spec(\Z[T])$ with $\mu$ corresponding to $\Z[T] \ra \Z[T_1,T_2]/ T \mapsto T_1 + T_2$.\\
$\mathbb{G}_m = Spec(\Z[T,T^{-1}])$ with $\mu$ corresponding to $\Z[T,T^{-1}] \ra \Z[T_1,T_1^{-1},T_2, T_2^{-1}]/ T \mapsto T_1T_2$.

\item \textbf{Group Scheme of an Elliptic Curve}: $E/K$ with $R$ (DVR) and good reduction at $\pp$. Find minimal equation $W(a_i \in R)$. Homogenize the equation and make it into a $R$-scheme $\EE$. Since addition on $E$ is by rational functions on $R$, we can define $\mu: \EE \times_R \EE \ra \EE$ as an $R$-morphism. The generic fibre is $E$.

\item \textbf{Group Scheme structure of $G(T)$}: Recall $G(T) = Hom_S(T,G)$, and define multiplictation (understood in the standard way) $\phi*\psi = \mu \circ \phi \times \psi \circ \delta_T$.  

\item \textbf{Translation by a morphism} in $G(S)$: $\tau_\sigma = \mu \circ (1 \times \sigma)$.\\

\item \textbf{Multiplication}: $[m]$ is defined inductively by $[1] = Id_G$ and $[m+1] = \mu \circ ([m] \times [1])$ and $[m-1] = \mu \circ ([m] \times i)$.  
 
\item \textbf{Neron Model of an EC}: $R$ DVR with $K = Frac(R)$. The neron model of $E/K$ is the !smooth group scheme $\EE/R$ (up to isomorphisms, not proper in general) that has $E/K$ as its generic fibre.\\
Must have \textbf{Neron Mapping Property}: \\
$\mathcal{X}/R$ smooth $R$-scheme with generic fibre $X/K$ and rational map $\phi_K: X/K \ra E/K$: \\
$\exists! R$-morphism $\phi_R: \mathcal{X}/R \ra \EE/R$

\item \textbf{Neron Models in Unramified Extensions}: $K'/K$ unramified and $R' = IC(R,K')$. If $\EE/R$ is the neron model of $E/K$, then $\EE \times_R R'/R'$ is the neron model of $E/K'$.

\item Important Case $X = Spec(K)$ and $\mathcal{X} = Spec(R)$. The neron property says the sections $\EE(R) \lra E(K)$ the $K$-points. \\

When $K = \ACK$: $\EE(R) = E(K) \Ra \EE/R$ is the Neron Model for $E/K$. 

\item \textbf{Theoretic Existence of Neron Models}: $R$ Dedekind, $K = Frac(R)$. Let $\mathcal{C}/R$ be a minimal proper regular model for $E/K$. Then the  largest subscheme $\EE/R$ of $\mathcal{C}/R$ is the neron model for $E/K$. 

\item \textbf{Minimal Models Settings}: $(R,\pp)$ DVR, residue field $k$, and $\mathcal{C}/R$ arithmetic surface (normal by definition), so it has Weil Irreducible Divisors: closed integral subscheme of dimension $1$. \\

For $x \in \mathcal{C}_\pp$ (the special fibre) and $\Gamma \in Div(\mathcal{C})$. A \textbf{uniformizer} for $\Gamma$ at $x$ is a function $f \in \OO_{\mathcal{C},x}$ that vanishes to order $1$ along $\Gamma$ and has no other zeros or poles in a nbhd of $x$. Recall $\OO_{\mathcal{C}, \Gamma}$ is a DVR containing $\OO_{\mathcal{C}, x}$. \\

\textbf{Local Intersection Index}:
\[(\Gamma_1 \cdot \Gamma_2)_x = dim_k \OO_{\mathcal{C},x}/(f_1,f_2)\]

\textbf{Fibral Divisor}: component of the special fiber. Extend to divisor with multiple terms. Their group is 
\[Div_\pp(\mathcal{C})\]
\textbf{Horizontal Divisor}: irreducible divisor which does not lie in the special fiber, then $\Gamma \ra Spec(R)$ will be surjective. 
\subsection{Tate's Algorithm}
\item \textbf{Tate's Algorithm} computes (1) Reduction type of the special fibre $\tilde{\mathcal{C}}/\overline{k}$. (2) Number of components $m = m(E/K)$ over $\overline{k}$ with multiplicity of the special fibre $\tilde{\mathcal{C}}$. (3) $v(\DD_{E/K})$. (4) Conductor Exponent $f = f(E/K)$. (5) Tamagawa Numbers or Fudge Factor: $c = c(E/K) = |\tilde{\EE}(k)/\tilde{\EE}^0(k)| = |E(K)/E_0(K)$.v \textbf{Ogg's formula}: $f(E/K) = v(\DD_{E/K}) - m(E/K) + 1$.\\
For the actual algorithm in general see Silverman2 or Cremona1 (for $\Q$). In order to obtain full information, it must be run for all primes dividing the Discriminant. Some transformations are required, Cremona actually sets them up, and Silverman does not because it is much more general.  

\end{enumerate}
\end{multicols}




\section{Modular Forms}
\textbf{Modular Forms for Dirichlet Charaters are preserved by Galois Action} 
\[M_k(N,\chi) \cong M_k(N,\sigma(\chi))\]


\begin{multicols}{2}
\begin{enumerate}
\item \textbf{Moduli}: collection of several algebraic structures. eg. All algebraic varieties; the modular curve whose points represent all isomorphism classes of EC. 

\item \textbf{Field of Definition} for $\{E\} \in ELL_\C$ is $K/\exists E_0 \in \{E\}: E_0/K$.\\
\textbf{Field of Moduli} for $\{E\} \in ELL_\C$ is $K/$ given $\sigma \in Aut(\C/\Q): E^\sigma \in \{E\} \Lra \sigma|_K = Id_K$. 
\item \textbf{Gamma Function}: $\Gamma(s/Re(s) > 0) = \int_{t = 0}^\infty e^{-t} t^s \frac{dt}{t}$, $\Gamma(1) =1$, $\Gamma(1/2) = \pi^{1/2}$, $\Gamma(s+1)=s\Gamma(s)$. 
\item \textbf{Riemann Zeta Function}: $\zeta(s/Re(s) > 1) = \sum_{n \geq 1} n^{-s} = \prod_p (1-p^{-s})^{-1}$. 
\item \textbf{Functional Equation}:
$\xi(s) = \pi^{-s/2} \Gamma(s/2)\zeta(s) \Ra \xi(s) = \xi(1-s)$.
\item \textbf{Mellin Transform:} $Mf(s) = \int_{t_0 > 0}^\infty f(it) t^s \frac{dt}{t}$.

\item \textbf{Theta Function}: $\Theta: \HH \ra \C/ \theta(\tau,l) = \sum_{n \in \Z^l} e^{\pi i}|n|^2 \tau$.

\item \textbf{Huewicz Thm}: $H_1(X,\Z) \cong Ab[\pi_1(X,x_0)]$. For a $g$-Torus, $\mathbb{T}_g \cong \bigoplus^{2g} \Z \gamma_j$. Moreover $H_1(X,R) = H_1(X,\Z) \otimes_\Z R \cong \bigoplus^{2g} R \gamma_j$. 

\item \textbf{Important Notation}: $\hh = \HH = \C_{Im>0}$; $\PP^1(\Q) = \Q \cup \infty$; $\hh^* = \hh \cup \Q \cup \infty$. Nbhds at $q \in \Q$ are semi-circles tangent to $\R$ at $q$.  
\[SL_2(\R) = \langle T = (1,1,0,1); S = (0,-1,1,0) | S^2 = (TS)^3 = (ST)^3 = -I\rangle\]
\[\gamma(z) = \frac{az+b}{cz+d} \Ra \gamma'(z) = \frac{1}{(cz+d)^2}\]
\textbf{Right Action}: $\{f:\hh \ra \C\}/GL_2(\R)$: 
\[f[\gamma]_k(z) = \det(\gamma)^{k-1} (cz+d)^{-k} f(\gamma(z))\]
$\Gamma$-invariance will preserve zeros and poles.\\
\textbf{Factor of Automorphy}: $j:GL_2(\Q) \times \HH \ra \C/ j(\gamma,\tau) = c_\gamma\tau + d_\gamma$. Satisfies $j(\gamma\gamma',\tau) = j(\gamma,\gamma'(\tau))j(\gamma',\tau)$
 
\item \textbf{Modular/Congruence (Sub)Groups}: $PSL_2(\Z) = SL_2(\Z)/\pm I$ 
\[\Gamma(N) \subseteq \Gamma \subseteq SL_2(\Z)\]
Principal Congruence Groups are: (index in $SL_2(\Z)$)
\[\begin{array}{ccc} 
 \Gamma(N) & \triangleleft \Gamma_1(N) & \triangleleft \Gamma_0(N)\\
((1,0)(0,1)) & ((1,*)(0,1)) & ((*,*)(0,*)) \\
 N^3\prod_{p|N}(1-p^{-2}) & & N \prod_{p|N}(1+p^{-1}) 
\end{array}\]
Always $\exists \min\{h/T^h = (1,h,0,1)\in \Gamma\}$. \\
SteinMF claims $[SL_2(\Z): \Gamma_0(N)] = \mu_0 = \prod_{p|N} \left(p^{v_p(N)} + p^{v_p(N)-1}\right)$. 

\item \textbf{Modular Curves}: $Y(\Gamma) = \hh/\Gamma$ and compactification or completion $X(\Gamma) = \hh^*/\Gamma = \overline{Y(\Gamma)}$ with quotient topology: Always Properly Discontinuous and Hausdorff, connected, compact Riemann Surface of Genus $g \in \N$. 

\item \textbf{Fundamental Domain} for the modular curve $Y(SL_2(\Z))$: $\mathcal{F} = \left\{\begin{array}{c} \tau / \\ |Re| \leq 1/2 \\ |\tau| \geq 1 \end{array}\right\} $ and $Y(SL_2(\Z)) \cong \mathcal{F} \cong \mathbb{S}^2\setminus N$ and $X(SL_2(\Z)) \cong \mathbb{S}^2$.

\item \textbf{Algorithm-Generators of Finite index Subgroup:} 
\[SL_2(\Z) = \bigsqcup^{< \infty} \Gamma r_i \textrm{coset reps}\] 
\[\Gamma r S = \Gamma \alpha_r \Ra s(r) = rS\alpha_r^{-1}\]
\[ \Gamma r T = \Gamma \beta_r \Ra t(r) = r\tau \beta_r^{-1}\]
\[\Ra \Gamma = \langle s(R) \cup t(R) \rangle\]


\item \textbf{Cusps} or \textbf{Parabolic Points} of the modular Curves are $\Gamma_S = \frac{\Q \cup \infty}{\Gamma}$ (always finitely many). \textbf{Parabolic Subgroup} $SL_2(\Z)_\infty = \{T^{j \in \Z}\} $ fixing $\infty$. Hence bijection with cusps:
\[\Gamma\setminus SL_2(\Z)/P \lra C(\Gamma);\,\,\,\Gamma \alpha P \lra \Gamma \alpha (\infty)\]


\item \textbf{Isotropy (Fixing) Group}: $\Gamma_\tau =\{\gamma \in \Gamma: \gamma\tau = \tau\}$. \\
\textbf{Elliptic Points:} Nontrivial isotropy group, always cyclic, always finitely many.\\
\textbf{Period of Elliptic Point}: $h_\tau = |\Gamma_\tau|/2$ or $|\Gamma_\tau|$ depending on whether $-I \in \Gamma_\tau$ or not. Notation: $\epsilon_2 = \#$ Elliptic points of order $2$: $SL_2(\Z)i$, $\epsilon_3 = \#$ of order $3$: $SL_2(\Z)\mu_3$, and $\epsilon_\infty = \#$ of cusps: $SL_2(\Z)\infty$, $d = deg(f: X(\Gamma) \ra X(1))$ 
\[g_{X(\Gamma)} = 1 + d/12 - \epsilon_2/4 - \epsilon_3/6 - \epsilon_\infty/2\] 




\item \textbf{Weaky Modular function}: Meromorphic at $\hh$ and $\Gamma$-invariant (action).\\
$SL_2(\Z)$-invariant enough $f(z+1) = f(z)$ and $f(-1/z) = z^k f(z)$. 

\item \textbf{Modular Function} or \textbf{Automorphic Forms}: ($A_k(\Gamma)$) Meromorphic at $\hh^*$ and $\Gamma$-invariant.\\
At $\infty$ means the corresponding $\mathbb{D}\setminus0 \ra \C$ extends to $0$.\\
$f \in \C[[q]]$ and $\forall n: |a_n| \leq Cn^r \Ra$ holomorphic at $\infty$.  

\item \textbf{Modular Form}: Holomorphic at $\hh^*$ and $\Gamma$-invariant.\\
Enough holomorphic at representatives of $\PP^1(\Q)$. 
\[M(\Gamma) = \bigoplus M_k(\Gamma)\]

\item \textbf{Cusp Form}: ($S_k(\Gamma)$): Modular Form and $f(\infty) = 0$ or $a_0 = 0$.
\[S(\Gamma) = \bigoplus S_k(\Gamma)\]
And holomorphic with simple zero at $\infty$:
\[\Delta_{E \in EC} \in S(SL_2(\Z))\]

\item \textbf{Cusps} $C(\Gamma)$: Orbits $\Gamma\setminus \PP^1(\Q)$. \\
One for $SL_2(\Z)$ and always finitely many for $\Gamma$. \\
\textbf{Width of} Cusp $\alpha = \gamma(\infty)$ for $\gamma \in SL_2(\Z)$: $\min\{h/ T^h \in \gamma^{-1}\Gamma \gamma\}$. Computed as: $\alpha = a/b \Ra \gamma = (a,b,c,d)/ ad - bc = 1$. Define $\delta(x) = \gamma (1,x,0,1) \gamma^{-1}$. Find minimal positive $x/ \delta(x) \in \Gamma$.
\[\begin{array}{ccccc}
\hh & \ra & \raa{f} & \ra & f(z) = \sum_{n \geq m} a_n q^n \\
q(z) = e^{2\pi i z}  & \searrow & & \nearrow & \in \C[[q]]\\
&& D\setminus \{0\} 
\end{array}\]

  
\item \textbf{Eisenstein Series}: $G_{k: even \geq 4}(z) \in M_k(SL_2(\Z))$
\[\sum^*_{(m,n)\in \Z^2}(mz+n)^{-k} = 2\zeta(k) + 2\frac{(2\pi i)^k}{(k-1)!} \sum_{n \geq 1} \sigma_{k-1}(n) q^n\]
Sum $t$-power divisors: $\sigma_t(n): = \sum_{1 \leq d|n} d^t$.\\
\textbf{Normalized} with $a_1 = 1$:
\[E_k(z) = \frac{(k-1)!}{2(2\pi i)^k} G_k(z) = -\frac{B_k}{2k} + q +  \sum_{n \geq 2} \sigma_{k-1}q^n\]

\item \textbf{Bernoulli Numbers}: (Fast Computable): $\frac{x}{e^x - 1} = \sum_{n \geq 0} B_n \frac{x^n}{n!}$ and $\zeta(k) = -\frac{(2\pi i)^k}{2\cdot k!}B_k$. $B_0 = 1$, $B_1 = -1/2$ and $B_{n:odd \geq 3} = 0$.\\
\textbf{Algorithm}: $K := \frac{2\cdot n!}{(2\pi)^n}$; $d:= \prod_{p-1|n}p$; $M = \left\lceil (Kd)^{1/(n-1)} \right\rceil$; $z = \prod_{p \leq M}(1-p^{-n})^{-1}$; $a = (-1)^{n/2+1}\lceil dKz\rceil$; $B_n = a/d$. 

\item \textbf{Valence Formula}: $ord_\infty(f) + ord_i(f)/2 + ord_{\zeta_3}(f)/3 + \sum^{*\neq i,\zeta_3}_{w \in F} ord_w(f) = k/12$.


\item \textbf{Main Result on Basis} For $SL_2(\Z)$: For $k = 2$ or odd: $M_k(SL_2(\Z)) = 0$. 
\[M_k (SL_2(\Z)) \cong S_k(SL_2(\Z)) \oplus \C G_k \cong \C[G_4,G_6]\]
i.e.: Basis $G_4^aG_6^b/4a+6b = k$
\[M_{k-12} \raa{\cong: \times \Delta} S_k\] 
\[dim M_k = \lfloor k/12 \rfloor + 1 - \delta_{k \equiv 2(12)} = dim(S_k) + 1\]

\textbf{Algorithm for basis of $M_k(SL_2(\Z)$ mod $q^n$}:
Compute $E_4,E_6$ (mod $q^n$). For $0 \leq b \leq \lfloor k/6 \rfloor$: if $a = (k-6b)/4 \in \Z$ compute $E_4^aE_6^b$ (mod $q^n$), saving all powers $m \leq a,b$.

\item \textbf{Victor Miller Basis} for $S_k(SL_2(\Z))$: $f_1,\ldots,f_d \in \Z[[q]]$ with $a_i(f_j) = \delta_{ij}$. Compute $F_4 = -\frac{8}{B_4}E_4$ and $F_6 = -\frac{12}{B_6}E_6$. Choose $0 \leq a,b \in \Z: 4a + 6b \leq 14$ and $4a + 6b \equiv k (12)$ with $a = b = 0$ when $k \equiv 0 (12)$. Compute $g_j = \Delta^j F_4^a F_6^{2(d-j)+b}$ and get Echelon form $f_i$. Can extend to basis for $M_k(SL_2(\Z))$ by normalizing $G_k$ with $a_0 = 1$ and subtract a linear combination of $f_1,\ldots,f_d$.



\end{enumerate}
\end{multicols}

\subsection{Moduli Spaces of Enhanced EC for $\Gamma$}
\begin{multicols}{2}
\begin{enumerate}
\item 
\[\frac{E/\C}{\textrm{Isomorphic}} \lra \frac{\HH}{SL_2(\Z)}\]
The $S(\Gamma)$ are the Moduli spaces and the $X(\Gamma)$ are the modular curves. The relations are always $\C$-isomorphisms preserving the extra structure.
\item For $\Gamma_0(N)$:
\[S_0(N) = \left\{E(\C), C = \langle e|e^N = 1\rangle\right\}/\sim = \left\{\left[E_\tau, \langle 1/N + \Lambda_\tau \rangle\right] \right\}\]  
\[S_0(N)  \lra  Y_0(N)/ [C, \langle 1/N+ \Lambda_\tau\rangle]  \lra  \Gamma_0(N)\tau\]
\textbf{Degree $k$ homogeneous functions}:
\[F: S_0(N) \ra \C\] 
\[\forall 0 \neq m \in \C: F(\C/m\Lambda, mC) = m^{-k} F(\C/\Lambda, C)\] 
\textbf{Dehomogenization of $F$}: 
\[f: \HH \ra \C/f(\tau) = F(\C/\Lambda_\tau, \langle 1/N + \Lambda_\tau \rangle)\]

\item For $\Gamma_1(N)$:  
\[S_1(N) = \left\{E(\C),Q/ Q^n = 0_E\right\}/\sim = \left\{[E_\tau, 1/N + \Lambda_\tau ] \right\}\]  
\[S_1(N)  \lra \lra  Y_1(N) /[C/\Lambda_\tau, 1/N+ \Lambda_\tau]  \lra  \Gamma_1(N)\tau\]
\textbf{Degree $k$ homogeneous functions}:
\[F: S_1(N) \ra \C\]
\[\forall 0 \neq m \in \C: F(\C/m\Lambda, mQ) = m^{-k} F(\C/\Lambda, Q)\]
\textbf{Dehomogenization of $F$}: 
\[f: \HH \ra \C/f(\tau) = F(\C/\Lambda_\tau, 1/N + \Lambda_\tau)\]

\item For $\Gamma(N)$:
\[S(N) = \left\{[E(\C),(P,Q)]/ E[N] = \langle P,Q \rangle\right\}/\sim\] 
\[S(N) = \left\{[E_\tau, (1/N + \Lambda_\tau, \tau/N + \Lambda_\tau)] \right\} \lra  Y(N) \]
\[[C/\Lambda_\tau, (1/N+ \Lambda_\tau,\tau/N+ \Lambda_\tau)]  \lra \Gamma(N)\tau\]
\textbf{Degree $k$ homogeneous functions}:
\[F: S(N) \ra \C\]
\[\forall 0 \neq m \in \C: F(\C/m\Lambda, (mP,mQ)) = m^{-k} F(\C/\Lambda, (P,Q))\] 
\textbf{Dehomogenization of $F$}: 
\[f: \HH \ra \C/f(\tau) = F(\C/\Lambda_\tau, (1/N + \Lambda_\tau, \tau/N + \Lambda_\tau))\]


\item \textbf{Modularity Theorem - Modular Parametrization:} $\forall E(\C)/j(E) \in \Q: \exists N: \exists$ surjective holomorphic  $X_0(N) \ra E(\C)$.

\item \textbf{Meromorphic Differentials}: $f(\tau)(d\tau)^{k/2}$ is $\Gamma$-invariant if $f$ is. 
\[\Omega(V \subseteq \C) = \bigoplus_{n \in \N} \Omega^{\otimes n}(V) = \{f_{\textrm{Mero}}(q)(dq)^n\} \]
Pullbacks: $\varphi^*(f\,(dq_2)^n) = (f\varphi) \varphi' (dq_1)^n$. \\
Meromorphic differentials on $X$ by compatible piecing. 
\[\omega: A_k(\Gamma) \raa{\cong} \Omega^{\otimes k/2} (X(\Gamma))\]
\end{enumerate}
\end{multicols}

\subsection{Eisenstein - Bernoulli}
\begin{multicols}{2}
\begin{enumerate}
\item See dimension formulas tables on DiamondMF Page 107-8, they agree with SteinMF.
\item Dimension formulas for $\Gamma_0(N)$.
\[\mu_0(N) = [SL_2(\Z):\Gamma_0(N)] = \prod_{p|N} p^{v_p(N)} + p^{v_p(N)-1}\]
\[\mu_{02}(N) = \prod_{p|N} (1+(-4/p)) \delta_{4\not|N}\]
\[\mu_{03}(N) = \prod_{p|N} (1+(-3/p)) \delta_{2,9\not|N}\]
\[c_0(N) = \sum_{d|N} \phi(gcd(d,N/d))\]
\[g_0(N) = 1 + \mu_0/12 - \mu_{02}/4 - \mu_{03}/3 - c_0/2\]
\[dim\,S_K(\Gamma_0(N)) = (k-1)(g_0 - 1) + (k/2-1)c_0 + \mu_{02}\lfloor k/4\rfloor + \mu_{03}\lfloor k/3 \rfloor\]
\[dim\,E_K(\Gamma_0(N)) = c_0 - \delta_{k = 2}\]
Commands: dimension$_-$cusp$_-$forms(Gamma0(N),k); dimension$_-$esi(Gamma0(N),k); dimension$_-$modular$_-$forms(Gamma0(N),k). 
\[dim\,S_k(\Gamma_0(N))_{new} = \sum_{M|N} \overline{\mu}(N/M) dim\S_K(\Gamma_0(M))\]
where $\overline{\mu}(R) = \prod_{p||R}(-2) \delta_{\not\exists q^3|R}$.
\[dim\,S_k(\Gamma_0(N)) = \sum_{M|N} \sigma_0(N/M) \, dim\,S_k(\Gamma_0(M))\]
Commands: dimnsion$_-$new$_-$cusp$_-$forms.

\item Dimension formulas for $\Gamma_1(N)$:
\[\mu_1(N) = \mu_0(N) \delta_{N = 1,2} + \phi(N) \mu_0(N)/2 \delta_{N \neq 1,2}\]
\[\mu_{12}(N) = \mu_{02}(N)\delta_{N < 4}\]
\[\mu_{13}(N) = \mu_{03}(N) \delta_{N < 4}\]
\[c_1(N) = \#Cusp(X_1(N)) = c_0(N) \delta_{N = 1,2} + 3\delta_{N = 4} + \sum_{d|N}\phi(d)\phi(N/d)/2 \delta_{N \neq 1,2,4}\]
\[g_1(N) = 1 + \mu_1/12 - \mu_{12}/4 - \mu_{13}/3 - c_1/2\]
where $g_1 = genus(X_1(N)) = dim\,S_2(\Gamma_1(N))$. 
\[\forall N \geq 3: dim\,S_K(\Gamma_1(N)) = a+1/2\delta_{N = 4, 2\not|K} + a + \lfloor k/3 \rfloor \delta_{N = 3} + a\delta_{other}\]
where $a(N,k) = (k-1)(g_1(N) - 1) + (k/2 - 1) c_1$.
\[dim\,E_k(\Gamma_1(N)) = c_1(N) - \delta_{k = 2}\]
\[dim\,S_k(\Gamma_1(N))_{new} = \sum_{M|N}\overline{\mu}(N/M) dim\,S_k(\Gamma_1(M))\]

\item Dimension formulas for Forms with Characters: Mod $N$ but conductor $c$.
\[dim\,S_k(N,\epsilon) - dim\,_{2-k}(N,\epsilon) =\]
\[(k-1)/12 \mu_0 - 1/2\prod_{p|N} \lambda(p,N,v_p(c)) \]
\[+ \gamma_4(k) \sum_{x \in A_4(N)} \epsilon(x) + \gamma_3(k) \sum_{x \in A_3(N)} \epsilon(x)\]
See definitions of $A_4,A_3,\gamma_4,\gamma_3$ in page 95-SteinMF.



\item  $dim(E_k(\Gamma(N))) = 1$: for $\overline{v} \in (\Z/N\Z)^2$ (column), $k \geq 3$: $G_k^{\overline{v}}(\tau) =$:
\[\delta_{\overline{c}_v=1} \left(\sum_{d \equiv d_v(N)}' d^{-k}\right) +\]\[ \frac{(-2\pi i)^k}{(k-1)!N^k} \sum_{n \geq 1}  \left(\sum_{\overset{m|n}{n/m \equiv c_v (N)}} sgn(m)m^{k-1}\mu^{d_vm}_N\right) q_N^n\]

\item For $\Gamma_0(N)$: 
\[M_2(\Gamma_0(N)) = S_2(\Gamma_0(N)) \oplus \C E_2(\Gamma_0(N))\] 
There is a generalized Eisenstein series $E_2(\Gamma_0(N))$ and
$dim(S_2(\Gamma_0(N))) = genus(X_0(N))$.

\item \textbf{Generalized Eisenstein on $\Gamma_1(N)$ and Dirichlet Char.} 
\[M_k,S_k,E_k(\Gamma_1(N)) = \bigoplus_\chi M_k,S_k,E_k(N,\chi)\]
where the space of eigenvectors: $M_k(N,\chi) = \{f \in M_k(\Gamma_1(N)): f[\gamma]_k = \chi(d_\gamma)f\}$.\\
Recall that $M_k,S_k,E_k(N,1) = M_k,S_k,E_k(\Gamma_0(N))$. 




\item Also $\mathbb{T}$-modules decomposition. 
\[M_k(\Gamma_1(N)) \cong E_k(\Gamma_1(N)) \oplus S_k(\Gamma_1(N))\] 
  
\item \textbf{Leopoldt - Generalized Bernoulli No} for $\chi \in D(N,\C)$:
\[\sum_{a = 1}^N \frac{\chi(a)\cdot x \cdot e^{ax}}{e^{Nx} - 1} = \sum_{k=0}^\infty B_{k,\epsilon} \cdot \frac{x^k}{k!}\]
\[\chi(-1) \neq (-1)^{k \geq 2} \Ra B_{k,\epsilon} = 0\]

\item \textbf{Algorithm - Generalized Bernoulli Numbers Algebraically:}  Want $B_{\leq k, \chi \in D(N,\C)}$.\\
Compute $e^{Nx} - 1 = \sum_{n \geq 1} N^n x^n/n!+O(x^{k+2})$, then its inverse, then multiplying by $x$: $g = x/(e^{Nx} - 1) \in \Q[[x]] + O(x^{k+1})$. 
\[f = \sum a_nx^n, f^{-1} = \sum b_n x^n \Ra b_n = -\frac{b_0}{a_0}\sum_{i=1}^{n}b_{n-i}a_i\]
$\forall 1 \leq a \leq N: e^{ax} = \sum_{n \geq 0} a^nx^n/n! + O(x^{k+1}) \Ra  f_a = g \cdot e^{ax} + O(x^{k+1})\in \Q[[x]]$;\\
$\forall j \leq k: B_{j,\epsilon} = j! \sum_{a=1}^N \epsilon(a) coeff_j(f_a)$.
Note: in steps 1 and 2, the arithmetic is in $\Q[[x]]$ rather than $\Q(\zeta_n)[[x]]$. If $k$ is large, we could compute $(e^{Nx}-1)^{-1}$ using asymptotically fast arithmetic and Newton iteration.

\item Sage:
G = DirichletGroup(N): computes the Dirichlet Group modulus N.\\
e = G.0: names e to the first element of G.\\
e(a): evaluates e at integer a.\\
e.bernoulli(j): finds $B_{j,e}$ in $\Q(\zeta_n)$.


\item \textbf{Generalized Bernoulli Numbers Analytically:}\\
Primitive $\chi \in D(f = N,\C)$. Then $\forall \sigma \in Gal(\overline{\Q}/\Q): \sigma(B_{n,\chi})= B_{n,\sigma(\chi)}$.\\
\textbf{Gauss Sum}: $\tau(\chi) = \sum_{r=1}^{f-1} \chi(r)\zeta_f^r$ \\
\textbf{Dirichlet Character L-function}: $L(s/ Re(s) > 1,\chi) = \sum_{n \geq 1} \chi(n) n^{-s} = \prod_{p:prime} (1- \chi(p)p^{-s})^{-1}$ extends to a meromorphic on $\C$.\\

Nonprincipal primitive $\chi \in D(f = N,\C)/\chi(-1) = (-1)^n$: 
\[L(n,\chi) = (-1)^{n-1}\frac{\tau(\chi)}{2} \left(\frac{2\pi i}{f}\right)^n \frac{B_{n,\overline{\chi}}}{n!}\]
Solve for Bernoulli numbers. 

\item \textbf{Denominator of Bernoulli Numbers}: \\
$d = 1$ if $f$ is divisible by two distinct primes or $2^{\mu > 2}$.\\
$d = 2$ if $f = 4$.\\
$d = np$ if $f = p > 2$.\\
$d = (1 - \chi(1+p))$ if $f = p^\mu$, $p > 2$, $\mu > 1$.\\
Then $dn^{-1}B_{n,\chi}$ is integral. 


\item \textbf{Explicit Basis for $E_k(\Gamma_1(N))$}: Primitive $\chi \in D(f = L,\C)$,$\psi \in D(f = R, \C)$: In $\Q(\chi,\psi)[[q]]$: Generalized Eisensten series:
\[E_{k,\chi,\psi}(q) = c_0 + \sum_{m \geq 1} \left(\sum_{n|m} \psi(n) \chi(m/n) n^{k-1} \right)q^m\]
\[c_0 = -\frac{B_{k,\psi}}{2k} \delta_{L = 1}\]

For $\chi \psi(-1) = (-1)^k$ except $(k,\chi,\psi) = (2,1,1)$: 
\[E_{k,\chi,\psi}(q^t) \in M_k(RLt, \chi/\psi)\]
For $RLt|N$, $E_{k,\chi,\psi}(q^t)$ form a basis for $E_k(N,\chi/\psi)$. \\
For $(k,\chi,\psi) = (2,1,1)$: 
\[E_{2,1,1}(q) - t \cdot E_{2,1,1}(q^t) \in M_2(\Gamma_0(t))\]
Both $E_{k,\chi,\psi}(q)$ and $E_{2,1,1}(q) - t \cdot E_{2,1,1}(q^t)$ with $t > 1$ are normalized eigenforms (eigenvectors for all $\mathbb{T}$) with eigenvalue for $T_m$ given by $\sum_{n|m} \psi(n) \chi(m/n) n^{k-1}$, and $\sigma_1(m)$ for $(m,t) = 1$ and $1$ for $m = t$, respectively.

\item \textbf{Algorithm - Enumerating Eisenstein Series}: Given: weight $k$, $\epsilon \in D(N,\C)$. Want: Basis of $E_k(N,\epsilon) \subseteq M_k(N,\epsilon)$ to precision $O(q^r)$.\\

(1) $k = 2$ and $\epsilon = 1 \Ra$ output $E_2(q) - tE_2(q^t)$ for each divisor $1 \neq t|N$ and terminate. \\
(2) $\epsilon(-1) \neq (-1)^k$ output empty list.\\
(3) Compute $G = D(N, \Q(\zeta_n))$ where $n = exponent(\Z/N\Z)^*$.\\
(4) Compute conductor of every element of $G$ (see previous algorithm)\\
(5) Form a list $V$ of all $\chi \in G: cond(\chi)cond(\chi/\epsilon)|N$.\\
(6) $\forall \chi \in V: \psi:=\chi/\epsilon$ and compute $E_{k,\chi,\psi}(q^t)$ mod $q^r$ for each $t|(N/cond(\chi)cond(\psi))$.\\


\item Sage: \\
E = EisensteinForms(Gamma1(N),k)\\

\end{enumerate}
\end{multicols}

\subsection{Modular Symbols and the Jacobian}
\begin{multicols}{2}
\begin{enumerate}
\item \textbf{Holomorphic maps between Tori} 
\[\varphi = (M,b): \C^g/\Lambda_g \ra \C^h/\Lambda_h/ M\Lambda_g \subseteq \Lambda_h\] 
\[\varphi(z + \Lambda_g) = M_{h \times g} z + b_{1 \times h} + \Lambda_h\]  
\textbf{Holomorphic Homomorphism}: $b = 0$


\item For $G = \Gamma_0(N)$: Choose $S_2(N)$ basis with $\Q$-Fourier Coefficients and 
\[S_2(N) = S_2(\Gamma_0(N)) = \bigoplus^g \C f_j = \bigoplus^{g} \R f_j \oplus \bigoplus^g \R i f_j\]
\[H_1(X_G,\Z) = Ab(\pi_1(X_G,\Z) \cong \bigoplus^{2g}\Z \gamma_j\]


\item \textbf{Period-Pairing}: Perfect, Nondegenerate, Hecke Invariant $\R$ or $\C$-linear Pairinig: $\langle T_nf,\gamma \rangle = \langle f, T_n\gamma \rangle$. 
\[S_2(G) \times H_1(X_G,\R) \ra \C\]
\[I_f(a,b) = \langle f, \gamma_{a-to-b}\rangle = 2\pi i \int_\gamma fdz = \int_a^b f(z) dz\]
\[\hat{S}_2(G) = Hom_\C(S_2(G),\C) \cong H_1(X_G,\R)\]

\item \textbf{Period Matrix}:
\[\Omega = (\omega_{j,k}) = \langle \gamma_j, f_k \rangle)_{2g \times g} = \left(\begin{array}{ccc} \int_{\gamma_1}f_1 & \ldots & \int_{\gamma_1} f_g \\ \vdots && \vdots \\ \int_{\gamma_{2g}} f_1 & \ldots & \int_{\gamma_{2g}} f_g\end{array}\right)\]

\item \textbf{Jacobian}:  Abelian variety $J(X_G)$ of dimension $g$: the $\R$-span of the 2g rows of the period matrix, over their $\Z$-span.
\[J(X) = \frac{\C^g}{\Lambda^g} \cong \frac{\textrm{Paths}}{\textrm{Loops}} = \frac{\Omega_{hol}^1(X)^{\wedge} \cong \bigoplus^{2g} \R\int_{\gamma_i}}{H_1(X,\Z) \cong \bigoplus^{2g} \Z \int_{\gamma_i}}\]

\textbf{Abel's Theorem}:
\[ Pic^0(X,x_0) \raa{\cong} Jac(X)/ [x] \mapsto \int_{x_0}^x\]


\item \textbf{Modularity Theorem}: $j(E) \in \Q \Ra \exists N: \exists$ surjective holomorphic homomorphism of complex tori: $J_0(N) \ra E$.\\

\item \textbf{Derived Maps} from Riemann Surfaces map $h: X \ra Y$: On Jacobians as Translation of differentials; On Picards as translation of meromorphic functions. $$h_J \circ h^J = [deg(h)]$$

 
%\[h^* = - \circ h: \C(Y) \ra \C(X)\]
%\[h^*: \Omega_{hol}^1(Y) \ra \Omega_{hol}^1(X)/ g(\tilde{q})d\tilde{q} = h^*g(q)h'(q)dq\]
%\[h_* = - \circ h^*: \hat{\Omega}_{hol}^1(X)  \ra \Omega_{hol}^1(Y)/\int_\alpha \mapsto \int_{h(\alpha)}\]
%\[norm_h:\C(X) \ra \C(Y)/ norm_hf(y) = \prod_{x \in h^{-1}(y)}f(x)^{e_x}\]
%\[tr_h: \Omega_{hol}^1(X) \ra \Omega_{hol}^1(Y)/ tr_h(\omega)|_{patch} = \sum_{i=1}^d (h_i^{-1})^*(\omega|_{{patch}_i})\]
%\[\int_{\delta} (h^{-1})^*\omega = \int_{h^{-1} \circ \delta} \omega\]
%\[\hat{tr}_h: \hat{\Omega}_{hol}^1(Y)^\wedge \ra \hat{\Omega}_{hol}^1(X)/ \int_\beta \mapsto \sum_{\textrm{lifts }\gamma} \int_\gamma\]
%\[h_* = (\hat{h}^*): H_1(X,\Z) \ra H_1(Y,\Z)/\int_\alpha \mapsto\int_{h \circ \alpha}\]
%$[H_1(Y,\Z): h_*(H_1(X,\Z))] < \infty \textrm{ and } [\Omega_{hol}^1(Y):V] < \infty \Ra [H_1(Y,\Z)|_V: h_*(H_1(X,\Z))|_V] < \infty$


\item \textbf{Forward Maps}:
\[\begin{array}{rcl c rcl}
Pic^0(X) & \raa{h_P} & Pic^0(Y) && [x] & \mapsto & [h(x)] \\
\da & & \da &\Ra& \da & & \da \\
Jac(X) & \raa{h_J} & Jac(Y) && \int_{x_0}^x & \mapsto & \int_{h(x_0)}^{h(x)}\\ 
\end{array}\]
\item \textbf{Backward Maps}:
\[\begin{array}{rcl c rcl}
Pic^0(Y) & \raa{h^P} & Pic^0(X) && [y] & \mapsto & [\sum_{x \in h^{-1}(y)} e_x x] \\
\da & & \da &\Ra& \da & & \da \\
Jac(Y) & \raa{h^J} & Jac(X) &&  \int_{y_0}^y & \mapsto & \sum_{x \in h^{-1}(y)} e_x \int_{x_0}^{x}\\ 
\end{array}\]


\item \textbf{Real Structure}: $G$ is of real type if $J = (-1,0,0,1)$ and $J^{-1}GJ = G$. Always true for $\Gamma_0(N), \Gamma_1(N)$. 

\item $*:z \mapsto z^* = -\overline{z}$ 
\begin{itemize}
\item On $H_1(X_G,\R)$: $\{a,b\} \mapsto \{a^*,b^*\}$ 
\[H_1(X_G,\R) = H_1^+(X_G,\R) \oplus H_1^-(X_G,\R)\] 
\[[H_1(X_G,\Z): H_1^+(X_G,\Z) \oplus H_1^-(X_G,\Z)] < \infty\] 
\item On $S_2(G)$: $f^*(z) := \overline{f(z^*)}$; $\gamma^* = J\gamma J = (a,-b,-c,d)$.  
\[f = \sum a_nq^n \lra f^* = \sum \overline{a_n}q^n\]
\[f^* \circ \gamma = (f \circ \gamma^*)^*\]
\textbf{Fixed part}: $S_2(G)_\R = S_2(G)^{(*)} =\bigoplus^g \R f_{i,\Q}$ ($\Q$-Fourier basis). 
\item On the pairing: 
\[\langle \gamma^*, f^* \rangle = \overline{\langle \gamma, f \rangle}\]
\[S_2(G)_\R \times H_1^+(X_G,\R) \ra \R\]
\[S_2(G)_\R \times H_1^-(X_G,\R) \ra i\R\]
\end{itemize}

\item \textbf{$2$-Modular Symbols}: $\{a,b\} \in \hh^{*2}$ representing $H_1(X_G,\R)$. 

\textbf{Manin-Drinfeld}: 
\[\forall G: \{a,b\} \in C(G)^2 \cong H_1(X_G,\Q)\]
\[\{a,\gamma a\} \in H_1(X_G,\Z)\]
Relations for $H_1(X_G,\Q)$: Torsion and 
\[\{a,b\} + \{b,c\} + \{c,a\} = 0 = \{ga,gb\} - \{a,b\}\]
Action:
\[GL_2(\Q)\setminus\M_2: g\{a,b\} = \{g(a),g(b)\}\]
Implications:
\[\{a, \gamma a\} = \{b, \gamma b\}\] 
\[\{a, \gamma \gamma' a\} = \{a,\gamma a\} + \{a,\gamma' a\}\] 

\item \textbf{THM}: Surjective Group Homomorphism: independent of $a \in \hh^*$:
\[G \hra H_1(X_G,\Z)/ g \mapsto \{a,ga\}\]
The kernel is generated by all commutators, all elliptic elements, all parabollic elements.

\item \textbf{2-Manin Symbols}: $\left\{(r_i)_G = r_i\{0,\infty\}_G\right\}_i$ where $SL_2(\Z) = \bigsqcup Gr_i$, $\mathcal{S} = \{r_i\}/r \in \mathcal{S} \Ra rTS \in \mathcal{S}$. Then  Also $SL_2(\Z)$-right action $(r)_G \cdot g =(rg)_G$.

\item \textbf{Equivalence of Coset reps}: $r = r'$ iff $N|c_1d_2-c_2d_1$ iff $(c_1:d_1) = (c_2:d_2)$. 

\item If $F$ is the fundamental domain for $SL_2(\Z)$ and $\mathcal{S}= \{r_i\}$, split it so $\mathcal{S} = \mathcal{S}' \sqcup \mathcal{S}'ST \sqcup \mathcal{S}'(ST)^2$ and 
\[F_G = \bigcup_{r \in \mathcal{S}'} \langle r \rangle := \{r(0), r(1), r(\infty)\}\]
Moreover, $r,r' \in \mathcal{S}$ are in the same orbit of $-T$ iff $r(\infty) = r'(\infty)$. So the number of $-T$ orbits is the number of $G$-cusps. And the length of the $-T$ orbit for $r$ is the width of the cusp $r(\infty)$. \\

For $G = \Gamma_0(N)$: the length of the $-T$ orbit $\{rT^n\}$ of $r = (c:d)$ is $N/gcd(N,c^2)$. 

\item \textbf{$M$-symbols} 
\[\mathbb{M}_2 \lra \PP^1(\Z/N\Z) = \{(c:d)/gcd(c,d,N) = 1\}/\sim\]
\[(r) = r\{0,\infty\} = \{b/d,a/c\} \lra (c_r:d_r)\]
Right action by $SL_2(\Z)$:
\[(c:d) \left(\begin{array}{cc} p& q \\ r & s \end{array}\right) = (cp+dr: cq+ds)\]
Relations:
\[(c:d) + (-d:c) = 0 = (c:d) + (c+d:-c) + (d:-c-d)\]


\item \textbf{Equivalence of Cusps}: $\alpha_j = p_j/q_j$/ $\exists u \in (\Z/N\Z)^*: q_2 \equiv uq_1$ and $up_2 \equiv p_1 (N)$. For $N = p$ prime: only two cusps: $m/n = 0$ iff $p\not|n$ and $m/n = \infty$ iff $p|n$. 

\item \textbf{Boundary Map}: On $H_1(X_G,\R)$ or Modular symbols: $\delta\{a,b\} = \{b\}-\{a\}$. On Manin Symbols and $M$-symbols: $\delta(r)_G = \delta(c:d) = \{a_r/c_r\} - \{b_r/d_r\}$. 

\item The SES:
\[0 \ra \mathbb{S}_2 \ra  \M_2 \raa{\delta} \mathbb{B}_2 = \bigoplus_{a \in C(G)}\Q \{a\} \ra 0\]
The Kernel of the $\delta$ map is the cuspidal modular symbols or $H(N)$ in the finite presentation, which should be $\Q^{2g}$-vectors but writing them over common denominator $d_2$, can be considered $\Z^{2g}/d_2$-vectors. 


\item \textbf{Manin Main Theorem - Finite Presentation}:
$H_1(X_G,\Q)$ is a quotient of $\bigoplus \Z(r_i)_G$.\\
\[\left\{0,\frac{b}{a}\right\} = \sum \left(\begin{array}{cc} p_k  & (-1)^{k-1} p_{k-1} \\ q_k & (-1)^{k-1} q_{k-1}\end{array}\right)\{0,\infty\}\]\[= \{0, \infty\} + \{\infty,0\} + \{0,-\}\ldots + \{-,\frac{b}{a}\}\]
where $p_{-2}/q_{-2} = 0/1$, $p_{-1}/q_{-1} = 1/0$. So the first two terms always cancel. In terms of $M$-symbols:
\[\{0,\alpha\} = \sum_{j=1}^k ((-1)^{j-1}q_j: q_{j-1})\]

\[\frac{\M_2 = \bigoplus_{r \in SL_2(\Z)/G} \Q(r)/(gr)-(r)}{B(G) = \langle (r)+(rTS)+r(TS)^2, (r) + (rS)\rangle}  \raa{\delta} \B_2(G)\]
\[\B_2 = \bigoplus_{a\in C(G)} \Q\{a\}\]
\[H_1(X_G,\Q) \cong H(G) = \frac{Z(G)=Ker(\delta)}{B(G)}\]

Relations on $M$-symbols: $(c:d) + (-d:c) = 0$ and $(c:d) + (c+d:-c) + (d:-c-d) = 0$ corresponding to the $B(G)$ relations above. 


Sage: \\
M = ModularSymbols(N,k = 2).manin$_-$generators(). lift$_-$to$_-$sl2z(N).\\
manin$_-$basis or [M.manin$_-$generators()[i] for i in M.manin$_-$basis()]: the first one returns list of indexes of sub-basis for the above relations.\\
$[$x.modular$_-$symbol$_-$rep() for x in M.basis()$]$: makes into modular symbols rather than Manin Symbols. \\
manin$_-$gens$_-$to$_-$basis: returns a matrix whose rows express each Manin symbol generator in terms of the subset of Manin symbols that form a basis. 

\item \textbf{Operators} $T_{p\not|N}$ on $\{a,b\}$: 
\[T_p\{a,b\} = (p,0,0,1)\{a,b\} + \sum_{0 \leq r < p}(1,r,0,p)\{a,b\}\]
\[T_pf(z) = pf(pz) + \frac{1}{p} \sum_{0 \leq r < p}f\left(\frac{z+r}{p}\right)\]
For $q|N$:
\[W_q = (q^ax,y,Nz,q^aw)/ \det = q^a\]
$\exists W_q-$involution on $H(N):=H_1(G,\Z)$ and on $S_2(N)$.\\
On $(c:d)$: In Sage $HeilbronnCremonaList(p)$: returns a list of matrices that allows to evaluate $T_p$ directly on $\PP^1(\Z/N\Z)$ rathern than on $\M_2$. 

\textbf{Fricke}: $W_N = \prod_{q|N}W_q$ corresponding to the transformation $(0,-1,N,0)$ or $z \mapsto -1/Nz$.\\ 


\item $*$ on $\PP^1(\Z/N\Z): (c:d) \mapsto (-c:d)$. Then $H^+(N) = H^+(X_G,\Q) = +1$-eigenspace. Then $H^+ = H/H^-$. It corresponds to $H_1(X_{\langle G,J \rangle},\Q)$.  Thus $H^+(N)$ needs only one extra relation from Manin-symbols $(c:d) = (-c:d)$. On the cusps $\S_2$, we need the extra relation $\{a^*\} = \{-a\}$.

\item The \textbf{Hecke Algebra} on $\Gamma_0(N)$ is $\mathbb{T} = \langle T_p, W_q\rangle_{p\not|N, q|N}$, where each operator is self-adjoint wrt Peterson Inner product on $S_2(N)$. So there are basis for $S_2(N)$ consisting of simultaneous eigenforms for all $T_p$ and $W_q$ with real eigenvalues. Similarly, the action $\mathbb{T}$ on $H(N) \otimes \R$ can also be diagonalized. \\
Also $*$ induces an involution on $H(N)$, which commutes with all the $T_p$ and $W_q$. Thus they preserve eigenspaces of $H^\pm(N)$, which are isomorphic $\T$-modules. \\

\item \textbf{Coefficients of eigenforms}: Normalize $a(1,f) = 1$: 
\[p\not|N \Ra f[T_p] = a(p,f)f\]
\[q |N \Ra f[W_q] = \epsilon_q f \Ra a(q,f) = -\epsilon_q \delta_{q^2\not|N}\]
Rest of Coefficients:
\[a(p^{r+1},f) = a(p,f)a(p^r,f) - \delta_{p\not|N}pa(p^{r-1},f)\]
\[(m,n) = 1 \Ra a(mn,f) = a(m,f)a(n,f)\]
\end{enumerate}
\end{multicols}


\subsection{Hecke Operators}
\begin{multicols}{2}
\begin{enumerate}

\item \[\begin{array}{ccc}
& \left(\frac{\C}{\Z+\lambda \Z}, \frac{\Z/nN + \Z\lambda}{\Z + \Z\lambda}\right) \\
&  = (E,C_{nN}) \\
&  \in X_0(nN)  \\
& \swarrow \pi_1\,\,\,\,\,\, \pi_2 \searrow \\
\left(\frac{\C}{\Z+\lambda \Z}, \frac{\Z/N + \Z\lambda}{\Z + \Z\lambda}\right) & & \left(\frac{\C}{\Z/n+\lambda \Z}, \frac{\Z/nN + \lambda\Z}{\Z/n + \lambda\Z}\right)\\
(E,C'_N) && (E/D_n, C_{nN}/D_n) \\
\in X_0(N)&& \in X_0(N)
\end{array}\]

\item \textbf{Hecke Operators}:  
\[f|T_{n,k} = \sum_{\overset{ad = n}{a \geq 1, 0 \leq b < d}}f[\gamma]_k\]
Preserving $WMf,Mf,M,S,M_k = S_k \oplus \C\cdot E_k$, so $E_k$'s are eigen-vectors for all $T_n$. \\
Satisfy the relation*:
\[T(p^{e+1}) = T(p)T(p^e) - pT(p^{e-1})\]

\item \textbf{Hecke on Fourier Expansions} Particular Groups:
Stein MF page 24: $\forall f \in M_k(SL_2(\Z)):$
\[ f|T_p = \sum_{n \in \Z} \left(a_{np} + p^{k-1}\delta_{p|n} a_{n/p}\right)q^n\]
\[\forall f \in M_k(SL_2(\Z)): f|T_n = \sum_{m \geq 0}\left(\sum_{1 \leq d|gcd(n,m)} \delta_{(d,N) = 1} d^{k-1} a_{mn/d^2}\right)q^m \]\[ a_1(f|T_n) = a_n(f)\]
\[f|T_n = \sum_{m \in \Z} \left(\sum_{1 \leq d|gcd(m,n)}d^{k-1}a_{mn/d^2}\right)q^m\]
From DiamondMF pg 171-2: $\forall f \in M_k(\Gamma_1(N)):$
\[ f|T_p(\tau) = \sum_{n \geq 0} \left((a_{np}(f) +  1_N(p)p^{k-1} a_{n/p}(\langle p \rangle f) \right)q^{n}\]
$\forall f \in M_k(\Gamma_0(N),\chi):$
\[f|T_p(\tau) = \sum_{n \geq 0} \left((a_{np}(f) +  \chi(p) p^{k-1} a_{n/p}(f) \right)q^{n}\]

\item \textbf{Hecke on Moduli Space $S_1(N)$}: $T_p: Div(S_1(N)) \ra Div(S_1(N))$ with $(E,Q) \mapsto \sum_C (E/C, Q+C)$ where $C$ is a $p$-subgroup of $E$ with $C \cap \langle Q \rangle = 0_E$.

\item \textbf{Matrix Hecke Operator} On Extension Victor Miller $M_k(SL_2(\Z)$-Basis. This leads to the study of the Hecke operator characteristic polynomial and its factorization.


\item \textbf{Maeda Conjecture}: The characteristic polynomial of $T_2$ on $S_k$ is irreducible for any $k$. Kevin Buzzard: several examples that the Galois group of the characteristic polynomial of $T_2$ is the full symmetric group (Buz96). \textbf{Improved Conjecture}: $\forall p$ primes and $k \geq 2$: The characteristic polynomial of $T_{p,k}$ acting on $S_k$ is irreducible. 



\item \textbf{Hecke on $\PP^1(\Z/N\Z)$ - Heilbronn Matrices}: 
\[(p,0,0,1)(c:d) = (c:pd) = (c:d)(1,0,0,p)\]
\[\{\infty,a\} = \sum_{k=-1}^k (r_j)_G\]
Define $M_i = (p,-r,0,1)r_iS$: 
\[(1,r,0,p)(c:d) = \sum_{i=0}^r (c:d)M_i\]
See Cremona Ch2pg22 for algorithm to compute Heilbronn matrices.

Sage: M = ModularSymbols(N,2), .boundary$_-$map(), .cuspidal$_-$submodule(), .matrix()\\
S = ModularSymbols(N,2).cuspidal$_-$submodule().matrix() and then S.T(p).matrix() will be a constant multiple of the identity, for $X_0(11)$: That constant is $p+1-\#E(\FF_p)$.

\item \textbf{Hecke Algebra - Temporary Notations for $\Gamma_1(N)$}: \\
Ring $\mathbb{T} = \langle T_n \rangle$ acting on$M_k(\Gamma_1(N))$. \\
\[\mathbb{T}' = Im(\mathbb{T}) \hra End(S_2(\Gamma_0(N))\]
\[\mathbb{T}'_\C = \mathbb{T}' \otimes_\Z \C = Span_\C(\mathbb{T}')\]
\[\tilde{\mathbb{T}}_\C: = \mathbb{T}'_\C \subseteq End(\C[[q]])\]
\[\mathbb{T}^{(N)} = \langle T_p \rangle_{p \not|N}\]

\item \textbf{Bilinear pairing}: 
\[\C[[q]] \times \tilde{\mathbb{T}}_\C \ra \C/\langle f, t \rangle = a_1(t(f))\]
\[S_2(\Gamma_0(N)) \times \tilde{\mathbb{T}}_\C \ra \C/\langle f,t \rangle = a_1(t(f))\] 
\[\langle f,\tilde{\mathbb{T}}_\C \rangle = 0 \Ra f = 0\]
2nd is Perfect:
\[\tilde{\mathbb{T}}_\C \cong Hom_\C(S_2(\Gamma_0(N)), \C)/ t \mapsto \langle -, t \rangle\]
\[\Psi:  S_2(\Gamma_0(N)) \cong Hom_\C(\tilde{\mathbb{T}}_\C,\C)/ f \mapsto \langle f, - \rangle\]

\textbf{THM}: $\forall \C$-linear $\varphi: \tilde{\mathbb{T}}_\C \ra \C$: 
\[f_\varphi:= \sum_{n \geq 1} \varphi(T_n) q^n \in \C[[q]] = \Psi^{-1}(\varphi) \in S_2(\Gamma_0(N))\]

\item \textbf{Basis for $S_2(\Gamma_0(N)$}: Computing anyway $\tilde{\mathbb{T}}_\C$, then computing a basis for $Hom_\C(\tilde{\mathbb{T}}_\C,\C)$.

\item \textbf{Algorithm Basis of $S_2(\Gamma_0(N))$} to precision $O(q^B)$:\\
(1) Compute $\mathbb{M}_2(\Gamma_0(N);\Q)$ (see above)\\
(2) Compute $\mathbb{S}_2(\Gamma_0(N);\Q)$ (see above)\\
(3) Set $d = \frac{1}{2} dim \mathbb{S}_2(\Gamma_0(N);\Q) = dim S_2(\Gamma_0(N))$\\
(4) Denoting $[T_n]$ as the matrix of $T_n$ acting on a basis of $\mathbb{S}_2(\Gamma_0(N);\Q)$ and $a_{ij}$ the function that returns the $ij$-entry. Compute to exhaust or to span a space of dimension $d$:\\
For $0 \leq i,j \leq d-1: f_{ij}(q) = \sum_{n=1}^{B-1} a_{ij}([T_n])q^n + O(q^B) \in \Q[[q]]$. These $f_{ij}$ are then a basis for $S_2(\Gamma_0(N))$. 



\item Sage: \\
\textit{M = ModularSymbols(N).basis()}\\
\textit{S = M.cuspidal$_-$submodule()}\\
Then compute a few Hecke operators matrices:
\textit{S.T(2).matrix()} = $c_2 \cdot$ Id\\
\textit{S.T(3).matrix()} = $c_3 \cdot$ Id\\
... Then $f_{0,0} = q + c_2q^2 + c_3q^3 + \ldots$.  For $11$ this is enough for a basis for $S_2(\Gamma_0(N = 11))$. \\
Actually, it does not always happens that we obtain a constant multiple of the identity, and we read one entry (the same) for several $S.T(n)$:\\
\textit{[S.T(n).matrix()[i,j] for n in range(B)]}\\

A more advance tool of Sage to do this:
\textit{M = ModularSymbols(N).basis()}\\
\textit{S = M.cuspidal$_-$submodule()}\\
\textit{f = S.q$_-$expansion$_-$cuspforms(B)}\\
Then $f(i,j)$'s will give us the $f_{i,j}$'s above. 

\item \textbf{$S_2(\Gamma_0(N))$ using eigenvectors}: Using modular forms that are common eigenvectors (called eigenforms) for $\mathbb{T}^{(N)}$, generated by $T_p$ with $p\not|N$, to construct a basis for $S_2(\Gamma_0(N))$ \\

For $M|N: \forall d$ divisor of $\frac{N}{M} \in \Z: \exists$ natural degeneracy map $\alpha_{M,d}: S_2(\Gamma_0(M))$. 


\item \textbf{$L$-series}: The Mellin Transform of $f$ is 
\[L(f,s) = \sum_{n \geq 1} a_n(f)n^{-s} =\]
\[ \prod_{p|N} (1-a_p(f)p^{-s} + p^{1-2s})^{-1}\prod_{p\not|N}(1-a_pp^{-s})^{-1}\]
These expression are obtain by substituting Fourier $f = \sum a_nq^n$ in the integral, and then applying relations of the coefficients obtain by $\mathbb{T}$. With additional factors: $\Lambda(f,s) = -\epsilon_N \Lambda(f,2-s)$ where $\epsilon_N$ comes from $f[W_N] = \epsilon_N f$. If $\epsilon = +1$, $L(f,1)$ is a zero of odd order, and if $\epsilon = -1$, $L(f,1)$ is a zero of even order. \\

Recall $\{0,\infty\} \in H_1(X_G,\Q)$ hence rational multiple of some period of $f$. Closing on that path, we obtain 
\[(1+p-T_p)\{0,\infty\} = \sum_{k=0}^{p-1}\{0,k/p\}\]
\[(1+p-a_p)\langle \{0,\infty\},f\rangle = \sum_k \langle \{0,k/p\}, f\rangle\]
Since $p\not|N$, then $\{0,k/p\} \in H_1(X_G,\Z)$, and RHS is a period of $f$. \\
If $\Omega_0(f)$ is the least positive real period of $f$, set $\Omega(f) = 2^{\delta_{rectangular\,period}}\Omega_0(f)$. Hence, $\Omega(f)/\Omega_0(f)$ is the number of components of the real locus of the elliptic curve $E_f$. 
\[\frac{L(f,1)}{\Omega(f)} = \frac{n(p,f)}{2(1+p-a_p)}\]
independent of $p$.
\end{enumerate}
\end{multicols}

\subsection{All Operators}
\begin{multicols}{2}
\begin{enumerate}
\item \textbf{Double-Cosets Ring}: $\Gamma \alpha \Gamma = \bigsqcup \Gamma \alpha_i$, and $\Gamma \beta \Gamma = \bigsqcup \Gamma \beta_i$. Then $\Gamma \alpha \Gamma \beta \Gamma = \bigcup \Gamma \alpha_i \beta_j$. Product:
\[(\Gamma \alpha \Gamma)(\Gamma \beta \Gamma) = \sum \#(\Gamma\alpha_i\beta_j = \Gamma \gamma) \Gamma \gamma \Gamma\]

\item \textbf{Double-Coset Operator}: $\Gamma_1\alpha\Gamma_2 = \bigsqcup_j \Gamma_1\beta_j \Ra f[\Gamma_1\alpha\Gamma_2]_k = \sum_j f[\beta_j]_k$.
\begin{itemize}
\item $\Gamma_1 \supseteq \Gamma_2$ and $\alpha = I$: $[\Gamma_1\alpha\Gamma_2]_k: M_k(\Gamma_1) \hra M_k(\Gamma_2)$.
\item $\alpha^{-1}\Gamma_1\alpha = \Gamma_2$: $[\Gamma_1\alpha\Gamma_2]_k: M_k(\Gamma_1) \raa{\cong} M_k(\Gamma_2)$.
\item $\Gamma_1 \subseteq \Gamma_2$ and $\alpha = I$ and $\Gamma_2 = \bigsqcup \Gamma_1 \gamma_j$: $[\Gamma_1\alpha\Gamma_2]_k: M_k(\Gamma_1) \ra M_k(\Gamma_2)$ projection (surjective) by symmetrizing over the quotient. 
\end{itemize}

\item 
\[\Delta = \{\det > 0\}\]
\[\Delta_0 = \left\{\det\left(\begin{array}{cc} (a,N) = 1 & * \\ c \equiv 0 & * \end{array}\right) > 0\right\}\] 
\[\Delta_1 = \left\{\det\left(\begin{array}{cc} a\equiv 1 (N)& * \\ c \equiv 0(N) & * \end{array}\right) > 0\right\}\] 

\item Correspondences on  $X \times Y$ as $Div(X) \ra Div(Y)$. 

\item \textbf{Diamond Operators}: 
\[M_k(\Gamma_1(N)) = \bigoplus_{\epsilon(-1) = (-1)^k} M_k(\Gamma_0(N),\epsilon)\] 
where the matrices acting are in $\Gamma_0(N)$ and the eigenvalue is $\epsilon(d_\gamma)$. 
\[\langle d\rangle_k: M_k(\Gamma_(N)) \ra M_K(\Gamma_1(N))\]
\[f \mapsto f|\left[\sigma_d \equiv (d^{-1},0,0,d)(N)\right] \]
Notice $\sigma_d \in \Gamma_0(N)$ and $f \in M_K(\Gamma_1(N))$. 




\item Consider $\hh, Y = \frac{\hh}{\Gamma} \cong Y' = \frac{\hh}{\alpha^{-1}\Gamma \alpha}, Y_\alpha = \frac{\hh}{\Gamma_\alpha = \Gamma \cap \alpha^{-1} \Gamma \alpha}$. Points correspond as $z \in \Gamma z \subseteq \hh, [z] = \Gamma z \in Y \cong \alpha^{-1}z \in \alpha^{-1}\Gamma \alpha z  \subseteq Y'; \bigcup [\epsilon_i z] \in Y_\alpha$ where $\Gamma = \bigsqcup \Gamma_\alpha \epsilon_i$.
\[\Gamma \alpha \Gamma = \bigsqcup \Gamma \alpha_i\]
\[\Gamma \alpha \Gamma \ra \tau_\alpha \in Corresp(X \times X)\]
\[\tau_\alpha([z] = \Gamma z) = \sum \left([\alpha_i z] = \Gamma \alpha_i z\right)\]


\item \textbf{All other operators}: 
\begin{itemize}
\item $T(n) = \sum_{\alpha \in \Delta_{\det = n}} [\Gamma \alpha \Gamma] \in R(\Gamma,\Delta)$. 
\item $T(a,d) = \Gamma \sigma_a(a,0,0,d)\Gamma$, with $a|d$, $(d,N) = 1$, $\sigma_a \equiv (a^{-1},0,0,a)(N)$. 
\item Then 
\[R(\Gamma,\Delta) = \Z[T(p),T(p,p)_{p\not|N}]\]
\[R(\Gamma,\Delta) \otimes \Q = \Q(T(n))\]
All $\Gamma \alpha \Gamma = T(m)T(a,d)$, which commute with each other.\\
If $(m,n) = 1$ or $m|N^\infty$ or $n|N^\infty$ then $T(mn) = T(m)T(n)$. 
\end{itemize}

\item $T(n)_k$ is $T(n)$ acting on $M_k(\Gamma_1(N))$.\\
$T(n)_{k,\epsilon}$ is $T(n)$ acting on $M_k(\Gamma_0(N),\epsilon) \subseteq M_k(\Gamma_1(N))$.\\
$T_0(n)_k$ is $T(n)$ acting on $M_k(\Gamma_0(N))$.

\item Relations as Endomorphisms: If $\mathbb{T}_N = \Z[T_p, S_p = T(p,p)_{p \not|N}]$ and $\mathbb{T}^{(N)} = \Z[T_{p\not|N}, S_p = T(p,p)_{p \not|N}]$
\[\langle T_n \curvearrowright M_k(\Gamma_1(N))\rangle = \langle T_p, \langle q \rangle_k \curvearrowright M_k(\Gamma_1(N))\rangle\]
\[= \left\{\begin{array}{cc} = \mathbb{T}_N \La k \geq 2 \\ \subseteq \mathbb{T}_N \La k = 1 \\
= \mathbb{T}_N = \Z[1/p]_{p\not|N} \La k = 0\end{array}\right\}\]
For $k = 1$, inclusion is replaced by equality after tensoring by $\Q$.\\
For $\mathbb{T}^{(N)}$, similar results hold but with all $p\not|N$. 

\item $\mathbb{T}^{(N)}$ eigenforms are $\mathbb{T}_N$ eigenforms iff their $\mathbb{T}^{(N)}$ eigenspace is $1$-dimensional. This holds for newforms.

\item $\mathbb{T}_N$ eigenspaces is at most $1$-dimensional. The collection of eigenvalues is contained in the ring of integers of a number field. 

\item Petterson Inner Product: always defined on cusp forms or pairs cups-modular forms. 
\[\langle f,g \rangle = \frac{1}{[\overline{\Gamma(1)}: \overline{\Gamma}]} \int_{D_\Gamma} f(z) \overline{g(z)} y^k \frac{dxdy}{y^2}\]

\item \textbf{Self-Adjoints}:
\begin{itemize}
\item $[\alpha]_k \ra [\alpha^i]_k$ where $\alpha^i\alpha = (\det\alpha)I$.
\item On $S_k(\Gamma_0(N)): T_n \ra T_n$.
\item On $S_k(\Gamma_0(N),\epsilon): T_n \ra \overline{\epsilon(n)}T_n$. 
\item On $S_k(\Gamma_1(N)): T_n \ra T_n \circ \langle \overline{n}\rangle$.
\item $S_k(\Gamma_1(N)) = \bigoplus_{\epsilon(-1) = (-1)^k} S_k(\Gamma_0(N),\epsilon)$ and each term decomposes into orthogonal $\mathbb{T}^{(N)}$-eigenspaces.
\end{itemize}

\item \textbf{W-operators}: 
\begin{itemize}
\item $w_N = (0,-1,N,0)$ and $[w_N]_k = [\Gamma_1(N) w_N \Gamma_1(N)]_k$. Then $W_N$ (notice the capital letter now): $W_N(f) = N^{1-k/2}f|[w_N]_k$. Then $W_N^2 = (-1)^k$. Also $W_N: (k,N,\epsilon) \ra (k,N,\overline{\epsilon})$. 

\item Now $Q|N$: $w_Q = (Qa,b,N,Qd)$ with $(N/Q)|(d-1)$. Then $W_Q: f \mapsto Q^{1-k/2}f[w_Q]_k$. On $M_k(\Gamma_1(N)): W_Q^2 = (-1)^k$. On  $M_k(\Gamma_0(N))$: preserves and $W_Q^2 = 1$. 
\item On $M_k(\Gamma_0(N))$: $W_N$ has two eigenspaces $\pm 1$ and the decomposition $S_k(\Gamma_1(N)) = E^+ \oplus E^-$ is $\mathbb{T}^{(N)}$ equivariant.  
\end{itemize}
\subsection*{NewForms and OldForms}
\item $dM|N$ and $i_d = (d,0,0,1)$. Then 
\[\iota_{d,M,N}^*: S_k(\Gamma_1(M)) \hra S_k(\Gamma_1(N))/ f \mapsto f[i_d]_k\]
is an injective $\mathbb{T}^{(N)}$-homomorphism.

\item \textbf{Old Space}: Span of images running over all $dM|N$ and $M \neq N$. \textbf{New Space}: Orthogonal Complement wrt Peterson Inner Product. 

\item $S_k(\Gamma_1(N))$ admits a basis consisting of $\mathbb{T}^{(N)}$ eigenforms where each is of the form $g|[i_d]_k$ for some $g \in S_k(\Gamma_1(M))^{new}$ and $g \in \mathbb{T}^{(M)}$-eigenform. 

\item Similar decomposition for $S_k(\Gamma_0(N),\epsilon)$ and $S_k(\Gamma_0(N))$ where the later corresponds to $\epsilon = 1$.   

\item \textbf{Atkin-Lehner - Multiplicity One Thm}: $f,g \in S_k(\Gamma_1(N)) \cap \mathbb{T}^{(ND)}$-eigenforms with the same eigencharacters: $\forall T: \theta_f(T)f = Tf$ and $\theta_f = \theta_g$. If $f \in S_k(\Gamma_1(N))^{new}$, then $g \in \C f$. 

\item On $S_k(\Gamma_1(N))^{new}$: $\mathbb{T}_N$-eigenform iff $\mathbb{T}^{(N)}$-eigenform iff $\mathbb{T}^{(ND)}$-eigenform. In general the last two are equivalent but they don't imply the first.  

 
\end{enumerate}
\end{multicols}

\subsection{Computing EC/N}
\begin{multicols}{2}
\begin{enumerate}

\item Compute $H^+(N) = H^+(X_G,\Q)$ as $M$-symbols: 
In general use $N\prod_{p|N}(1+p^{-1})$ symbols: $(0 \leq c < N:1)$, $(1:0 \leq d < N)$ with $gcd(d,N) = 1$, and some $(c:d)$ with $1,N\neq c|N$ and $gcd(c,d)=1, gcd(d,N) > 1$. \\

For $N = p$, use $p+1$ symbols $(c) = (c:1)$ and $(\infty) = (1:0)$. Relations $(c) + (-c) = 0$ and $(c) + (-1-c^{-1}) + (-(c+1)^{-1}) = 0$.  



As tables, considering relations, in terms of generators and with $\Z$-linear combinations and common denominator $d_1$, to convert them. Compute $\delta$. No need list of cusps, only some and testing of equivalence. Write $\delta \equiv (\delta_{ij})$ and $Ker(\delta) = H(N)$, written as $\Z$-vectors with common denominator $d_2$.

\item Compute enough Hecke operators $T_{p\not|N}$ and $W_{q|N}$ on $H(N)$ written as $\Z^{2g}$-vectors. 
\[a(p,f) f = a(1,f)T_pf\]
\[a(q,f) = -\underbrace{\epsilon_q}_{\pm 1} \delta_{q^2\not|N}/\,\,f[W_q] = \epsilon_q f\] Look for $1$-dim eigenspaces and $\Q$-eigenvalues.\\

Alternatively, working in $H^+(N)$: consider extra relations $(d:c) = -(c:d)$, matches $4$ M-symbols at once rather than $2$. And computing $ker(\delta)$ we use $\alpha \equiv \pm \beta (\Gamma_0(N))$.

\item \textbf{Splitting Off $1$-dim eigenspaces}:
\[g \in S_2(M)^{new}; g[W_l] = \epsilon g; N = q^\beta M\]
Then the oldclass determined by $g$ is 
\[Span(g_i = q^ig(q^iz); 0 \leq i \leq \beta\]
When $l \neq q: g_i[W_l] = \epsilon g_i$ and when $l = q: g_i[W_l] = \epsilon g_{\beta - i}$. \\
In general, $N/M = \prod q^\beta$: $g \in S_2(M)^{new}$ spans an old class of dimension $\prod (\beta+1)$.
\[n_i^+ \frac{1}{2}(\beta_i + 1 + \delta_{\epsilon_i = +1} - \delta_{\epsilon_i = -1}\]
\[n_i^+ + n_i^- = \beta_i+1\]

\item \textbf{Rational New Forms}: 
\[S_2(N) = S_2(N)_\Q \otimes_\Q \C\]
\[S_2^{old}(N) = Span_\C\{g(Dz)/ g \in S_2(M|_{<}N), D|(N/M)\} \subseteq S_2(N)\] 
\[S_2^{new}(N) = S_2^{old}(N)^\bot \subseteq S_2(N)\] 
wrt Petersson inner product. Then $\T|_{S_2^{new}(N)}$ is semisimple and $S_2^{new}(N)$ has a basis consisting of simultaneous eigenforms called newforms.\\

In general, newforms come in conjugate sets of $d \geq 1$ whose eigenvalues generate an algebraic number field of degree $d$. The periods of such a set of conjugates $\{f\}$ form a lattice $\Lambda$ of rank $2d$ in $\C^d$, and thus an abelian variety $A_f = \C^d/\Lambda$ defined over $\Q$. Interest in $d = 1$.\\

\item \textbf{Computing Fourier Coefficients}: 
\[\forall p\not|N: \frac{L(f,1)}{\Omega(f)} = \frac{n(p,f)}{2(1+p-a_p)}\]

If $L(f,1) \neq 0$ and having $n(p_0,f)$ for the minimal prime $p_0\not|N$: we only need $n(p,f)$ for $p \neq p_0$. Recall:
\[(1+p-a_p)\langle \{0,\infty\},f\rangle = \sum_{k=0}^{p-1} \langle \{0,k/p\},f\rangle \in \R\]
Then express the RHS in terms of the $M$-symbols of $H^+(N)$. ANd then project onto the one-dimensional subspace of $f$, here we take the dot product with the dual eigenvector normalized to have relatively prime coefficients in $\Z$. Big scale factor $d_1d_2$. \\

If $L(f,1) = 0$, then choose $n/d$ with $(d,N) = 1$
\[(1+p-T_p)\{n/d,\infty\} = \{0,pn/d\} + \sum_{k=0}^{p-1}\{0,(n/d+k)/p\} - (p+1)\{0,a\} \in H_1(X_0(N),\Z)\]
Then express the RHS in terms of the $M$-symbols of $H^+(N)$ and proceed as above. 
\[\frac{Re(\langle \{n/p, \infty\}, f \rangle)}{\Omega(f)} = \frac{n(n/p,p,f)}{2(1+p-a_p)}\]

\item \textbf{Compute Periods}: Work in $H(N) = H_1(X_G,\Z) \otimes \Q = \bigoplus \Q \gamma_i$ as columnd vectors and dual as row vectors. For each $f$: compute normailized $v^\pm$ dual (primitive) eigenvectors with $\pm$-eigenvalues resp, and $\gamma^\pm \in H^\pm(N) \cap \Q^{2g}$ eigenvecors. 
\[x = \langle \gamma^+, f \rangle, y = -i\langle \gamma^-,f\rangle \in \R\]
\[\langle \gamma_j,f \rangle = \underbrace{a_j}_{v^+\gamma_j}x + \underbrace{b_j}_{v^-\gamma_j}yi\]
\[\Z^2 \cong \Lambda_j = Span_\Z(a_j,b_j)_{2g} = \langle (\lambda_1,\mu_1), (\lambda_2,\mu_2) \rangle\]
Type 1: $v^+ \equiv v^- (2)$, then $(\lambda_1,\mu_1) = (2,0), (\lambda_2,\mu_2) = (1,1)$ and $\omega_1 = 2x, \omega_2 = x+iy$ and $\Omega(f) = \Omega_0(f)$ and $\Delta(E_f) < 0$. \\
Type 2: $v^\pm$ are independent mod(2), then $(\lambda_1,\mu_1) = (1,0), (\lambda_2,\mu_2) = (0,1)$ and $\omega_1 = x, \omega_2 = iy$ and (rectangular) $\Omega(f) = 2\Omega_0(f)$ and $\Delta(E_f) > 0$. \\
Note $uv^\pm, \gamma^\pm/u, u\lambda_j, u \mu_j$. \\


Tingley: Compute $\langle \gamma,f\rangle$ such that $v^\pm \gamma \neq 0$ (sometimes two periods $\gamma$). Then $I_f(\alpha) = I_f(\alpha,\infty)$ and  Then homomorphism $P_f: G \ra (\C,+)/ g \mapsto P_f(g) = = I_f(\alpha,g(\alpha)) = I_f(\alpha) - I_f(g(\alpha))$, is such $\Lambda_f = Im(P_f)$.\\

To Compute $P_f(a,b,cN,d)$ use $\alpha = \frac{-d+i}{cN}$ and $M(\alpha) = \frac{a+i}{cN}$. 






\item Compute by duality the rational $S_2^{new}(N)_\Q$, recognize old forms because they were already found.
\item Find $\Z$-basis for $\Lambda_f$ by computing the generating periods to high precision. 
\item Give $\Z$-basis for $\Lambda$, compute the coefficients of $E_f$.
\item Compute other stuff. 

\end{enumerate}
\end{multicols}


\section{Setzer - EC on Complex Quadratic Fields}
Are there any EC$/K$ with good reduction everywhere. Tate unpublished examples.
\begin{multicols}{2}
\begin{enumerate}

\item Thm1: $E/K$ and $W = gcd(\Delta)$ over all integral models of $K$. Then there is ideal class $C$ such that over all integral models the discriminant is $WB^{12}$ where $B$ runs over all integral ideals $B \in C$. \\

\textbf{Proof}: 
Start with $E/ \Delta = WB^{12}$ and assume $B = uB'$ with $B'$ another integral ideal on the same class. Then $B = uB'$, i.e. $\Delta = Wu^{12}B'^{12}$.\\

If $\beta^{e > 0}||u \Ra \beta^{e}|B \Ra \beta^{12e}|\Delta$. Using a trasnformation $T_1 = T(u_1,r_1,s_1,t_1)$ with $\beta^{d \geq e} |u_1$. Then the new model is integral and $\Delta = WB^{12}/u_1^{12} = WB''^{12}$ where $\beta\not| B''$. The choice of the $r_1,s_1,t_1$ will have to be restricted to certain residue classes modulus $\beta^{6e}$. 

\item Cor1: $E/K$ with good reduction everywhere. Then $\exists$ ideal class $C$ such that all integral models of $E$ have discriminant $B^{12}$ with $B$ integral $\in C$. 

\item Cor2: If $gcd(h_K,6) = 1 \Ra E$ has a global minimal model. 

\item Thm2: $E/k = \Q[\sqrt{-m}]$ and let $K$ be the normal closure of $2$-vidision field of $E$ over $\Q$ and $G = Gal(K/\Q)$. \\
(1) $K|L|\Q$ with $[L:\Q] = 3 \Ra L$ not Galois over $\Q$. Ramify primes in $L/\Q$ are those dividing $2m$. The only prime which can triply ramify in $L/\Q$ is $2$.\\
(2) $G \cong S_3, S_3 \times \Z/2\Z, S_3 \times S_3$. \\
(3) If $G = S_3 \times \Z/2\Z$, then $\exists$ subgroup $S_3$ and corresponding subfield $F/\Q$ totally real. \\
(4) If $G = S_3 \times S_3$, there are two subgroups $S_3$ and corresponding subfields $F_i/\Q$, one of them totally real and one totally complex. 

\item Any $S_3$ extension of $\Q$ contains a quadratic extension called the Discriminant Extension. 

\item Thm3: $E/k = \Q[sqrt{-m}]$ of good reduction everywhere and $E(k)[2] \neq \emptyset$ iff $m = 65m_1$, $m_1$ is square both mod $5$ and $13$, and $65$ is square mod $m_1$. 
Can take $\Delta = D$ odd and changing to $y^2 = x^3 + Ax^2 + Bx$ with new $\Delta = 2^{12}D$ and $D$ coprime to $2$ and integral. Then $-16B^2(A^2-4B) = 2^{12}D$. By Thue finitely many solutions for a given $D$ and by Thm1, finitely many $D$'s, so can be checked for a given field. 

\item Thm4: List of primes where No $E/\Q[\sqrt{-m}]$ with good reduction everywhere.\\
There are $8$ elliptic curves isomorphism classes $E/\Q[\sqrt{-65}]$ all with $2-K$-point. 

\item Thm5: $k = \Q[\sqrt{-m}]$ with $(h_k,6) = 1 \Ra$ No $E/k$ having good reduction everywhere. 
\end{enumerate}
\end{multicols}


\section{BSD - Notes on Elliptic Curves}
\begin{multicols}{2}
On curves $\Gamma: y^2 = x^3 - Ax - B$.
\begin{enumerate}
\item Let $\mathcal{A}$ be the $\Q$-points of $\Gamma$. $g$ the number of generators of infinite order and $g_2$ the number of generators of even order, which is $0,1,3$ depending on the rational roots of the RHS. Then $\#\mathcal{A}/2\mathcal{A} = 2^{g + g_2}$. 


\item $2$-Covering $\exists D/\Q$ curve with commutative diagram $(D \lra \Gamma) \ra \Gamma$ where $\Gamma \raa{[2]} \Gamma$ and generic points $(X \lra x) \ra 2x$. The covering classes where $(X \lra X')$ correspond to $x \lra x + \delta$ with $2\delta = 0$ are an abelian group under composition of coverings. All elements have order 2 but the identity.

\item $G = 2$-coverings with a point in each $p$-adic $= 2^k$ with $\#G = 2^k$. $G_2 = 2$-coverings with a $\Q$-point. Then $G' \cong \mathcal{A}/2\mathcal{A}$ and $\#G = 2^{g + g_2}$

\item $2$-coverings with $\forall p: D$ has $p$-adic point have a positive rational divisor of degree $2$.\\
Recall for EC $l(K_c) = 1$ and $deg(K_c) = 0$ and RRT says $l(D) - l(K_c-D) = deg(D)$ and if $deg(D') < 0 \Ra l(D') = 0$.

\item The $2$-coverings are represented by curves $D: y^2 = g(x) = (a,b,c,d,e)(x)$ quartics with invariants $I = 12ae - 3bd + c^2$ and $J = 72 ace + 9bcd - 27ad^2$. \\

Given invariants $I,J$:  $\Gamma: y^2 = x^3 - 27Ix - 27J$. And given $\Gamma: y^2 = x^3 - 27Ix - 27J$, the invariants of the covers are $I\lambda^4, J \lambda^6$. 


\item Reduced Cover $y^2 = g(x)$: (1) $a,b,c,d,e \in \Z$ and $a \neq 0$. (2) $\not \exists p > 3: p^4|I$ and $p^6|J$. (3) Not true $3^5|I, 3^9|J$ nor $3^4||I, 3^6||J, 3^{15}|4I^3 - J^2$. Not true $2^6|I, 2^9|J, 2^{10}|8I+J$. (4) $|b| \leq 2|a|$. \\
Finally, if $g(x)$ has $n = 0,2,4$ roots:\\
$n = 0 \Ra a > 0$ and $H \leq 2K + 2K^{1/2}\varphi^1 - 6K^{1/2}a - 2a\varphi_1$. \\
$n = 2 \Ra H \geq 9a^2 - 2a\varphi + \frac{1}{3}(4I - \varphi^2)$.\\
$n = 4 \Ra H \geq 4a\varphi^2 0 \frac{4}{3}(I 0 \varphi_2^24$. \\
where $\varphi^3 - 3I\varphi + J = 0$ and the roots are $\varphi_1 > \varphi_2 > \varphi_3$; $H = 8ac - 3b^2$; $K = \frac{1}{3}(4I - \varphi_1^2)$. 

\item \textbf{Main Theorem}: All nontrivial $2$-covering of $\Gamma$ in $G$ can be represented by a reduced curve $D: y^2 = g(x)$, and the collection of such is finite and computable. 

\item To study the local $p$-adic solubility of $y^2 = g(x)$: write $g_1(x) = x^4g(x^{-1})$. Given $p, x_0$:
\[p^\lambda||g(x_0), p^\mu||g'(x_0), p^\alpha||4I^3 - J^2\]
And see if we can find $x$ near $x_0$ making $g(x)$ a $p$-adic square. 

\item \textbf{Solubility}: \\
For $p$ odd. $p^\nu|(x-x_0)$: If $g(x_0)$ is a $p$-square or $\lambda - \mu \geq \nu > \mu$ CAN! If $\lambda \geq 2 \nu$ and $\mu \geq \nu$ MAY! Otherwise $y^2 = g(x)$ cannot be solved $p$-adically.\\
For $p = 2$. $2^\nu|(x-x_0)$: if $g(x_0)$ is a $2$-square or $\lambda - \mu \geq \nu > \mu$ or $\nu > \mu, \lambda = \mu + v - 1$ even or $v > \mu, \lambda = \mu + \nu - 2$ even and $2^{-\lambda} g(x_0) \equiv 1(4)$ CAN! If $\lambda \geq nu, \lambda \geq 2\nu$ or $\lambda \geq \nu$ and $\lambda = 2\nu - 2$ and $2^{-\lambda}g(x_0) \equiv 1(4)$ MAY! Otherwise NOT. 

\item Remarks: No algorithm to decide whether the covering has a $\Q$-point. 
\end{enumerate}
\end{multicols}

\section{Cassels - Arithmetic on Curves of genus 1 I}
\begin{multicols}{2}
Curves $x^3 + y^3 + dz^3 = 0$
\[M = \{\tilde{m} \in K^*/K^{*3}: \exists (x,y,z) \in K: m^{-1}x^3 + my^3 + dz^3 = 0\}\]
\[M^1 = \{\tilde{m} \in K^*/K^{*3}: \forall p: \exists (x,y,z) \in K_p: m^{-1}x^3 + my^3 + dz^3 = 0\}\]
\[M^2 = \{\tilde{m} \in M: \exists \Delta \in K(m^{1/3}): Norm(\Delta) = d, \forall p: \exists (u,v,w) \in K_p: f_\Delta(u,v,w) = 0\}\]
\[M,M^2 \subseteq M\]
\begin{enumerate}

\item Basically solving $Norm(\frac{m^{-1/3}x + m^{1/3}y}{-z}) = d$. Pick first $Norm(\Delta) = d$ and $\Delta \Theta^3 = Xm^{-1/3}+Ym^{1/3}$. If $\Theta \in K(m^{1/3}) \Ra \Theta = um^{-1/3} + v + wm^{1/3}$, then $0 = Spur(\Delta\Theta^3) \Ra f_\Delta(u,v,w) = 0$.

\item $\exists U: M^1 \times M^1 \ra \mu_3$ multiplicative on both entries, skewsymmetric $U(a,b)U(b,a) = 1$.
\[M^2 = \{m \in M^1: U(m,\cdot) = 1\}\]
Then the difference on number of generators of $M^1$ and $M^2$ is even (proven.) Strongly that of $M$ and $M^2$ is also even (not proven.)

\item Assuming $C(\ACK)[n] \subseteq C(K)$. Then
\[h: C(K)/nC(K) \lra H^0(n) \subseteq H(n) = \prod^{n^2-1} K^*/K^{*n}\]
\[a \mapsto \prod_{C(\ACK)[m]^*} f_d(a)\]
where $div(f_d) = n(d) - n(\infty = 0_E)$. 
Equivalent definitions $C(K_p), H_p^0(n), H_p(n)$ for the local $p$-adic points and $n$-powers classes. 

\item $K \hra K_p$ induces a mapping $H(n) \ra H_p(n)$.
\[H^0(n) \subseteq H^1(n):=\{x \in H(n) \mapsto H_p^0(n), \forall p\} \subseteq H(n)\]
Then $H^1(n)$ is finitely generated. \\

Rmk: Weil made the assumption $C(\ACK)[n] \subseteq C(K)$ because he only wanted to prove finitely generated, so extending the ground field doesn't make a difference. 

\item Curve $C: x^3 + y^3 + dz^3 = 0$. For points $X_1 + X_2 + X_3 = 0$ obtained by intersection with $lx + my + nz$, we have 
\[f = \frac{lx + my + nz}{x + y} \Ra div(f) = (X_1) + (X_2) + (X_3) - 3(\infty)\]
\[X = (x,y,z) \Ra -X = (y,x,z); \zeta_3 X = (x,y, \zeta_3 Z)\]

The $0_C = (1,-1,0)$.
The Sum is given by
\[h \left(\begin{array}{c} x \\ y \\ z\end{array}\right) = 
\left(\begin{array}{c} b \\ c \\ a\end{array}\right)
\left(\begin{array}{c} c\\ a \\ b\end{array}\right)
\left(\begin{array}{c} e \\ f \\ g \end{array}\right)^2 -
\left(\begin{array}{c} f \\ g \\ e \end{array}\right)
\left(\begin{array}{c}g \\ e \\ f \end{array}\right)
\left(\begin{array}{c}a \\ b \\ c  \end{array}\right)^2\]
Using it we can compute $-X = (b,a,c)$ (recall they are ratios).

The formulas that $\tau(x,y,z) = (\rho^2x^3 + \rho y^3 + dz^3, \rho x^3 + \rho^2y^3 + d^3, - 3xyz)$ come from using the formulae above and then multiplying by $\rho^2(1-\rho) = \rho^2 - 1$ to complement the factor in the $z$ variable and make it $\tau^2 = -3$. 

\item 

\item $\mathcal{D} = K \oplus K\delta + K\delta^2$ where $\delta^3 = d$. Then the morphisms $\delta \mapsto \delta \zeta^i$ with $i = 0,1,2$. And the norm $Norm(a+b\delta + c\delta^2) = a^3 + b^3 d + c^3 d^2 - 3abcd$. 


\end{enumerate}
\end{multicols}

\section{Cassels - AC g=1 II}
\begin{multicols}{2}
In the goal of finding generators of $EC$ (rank and torsion).
Cassels remarks $C$ is the jacobian of a $\nu$-covering curve $D$?
\begin{enumerate}
\item $\nu$-Covering $U$ denoted $D(U)_\nu$:
\[\begin{array}{lll} C & \raa{\nu} & C \\\updownarrow/\ACK & \nearrow/K \\ D
\end{array} \textrm{generic}
\begin{array}{lll} x_1 & \raa{\nu} & \nu x_1 = x_0 \\\updownarrow & \nearrow \\ X \end{array}\]
\item $\nu$-Covering equivalence:
\[\begin{array}{lll} D & \lra/\ACK & C \\\updownarrow & & \updownarrow \\ D' & \ra & C \end{array} \textrm{generic} \begin{array}{lll} X & \lra/\ACK & X' \\\updownarrow & & \updownarrow \\ x_1 & \ra & x_1 + \delta (\nu \delta = \infty) \end{array} 
\]
\item Behavior of maps for any other point $\af \in C$:
\[\textrm{generic} \begin{array}{lll} x_1 & \raa{\nu} & x \\\updownarrow & \nearrow \\ x_1 - \af_1 \end{array} \textrm{points} \begin{array}{lll} \af_1 & \raa{\nu} & \af \\\updownarrow & \nearrow \\ \infty \end{array}\]

\item Survive \textbf{First Descend}: $M^{(1)} = \{(D,U,\nu)/ \forall \pp \exists \mathbb{U} \in D(U)(K_\pp)\}$.

\item $\nu^2$-Covering $S$ denoted $D(S)_{\nu^2}$:
\[\begin{array}{lll} C & \raa{\nu^2} & C \\ \updownarrow/\ACK & \nearrow/K \\ E
\end{array}\]

\[ \begin{array}{lllll} & \textrm{extend}\\ C & \raa{\nu} & C & \raa{\nu} & C \\\updownarrow/\ACK & & \updownarrow/\ACK & \nearrow \\ E & \ra/K & D \end{array}  \begin{array}{lllll} & \textrm{gen} \\ x_2 & \raa{\nu} & x_1 & \raa{\nu} & x \\\updownarrow & & \updownarrow & \nearrow \\ Y & \ra & X \end{array} 
\]

\item $\forall (D,U,\nu) \ni \mathbb{U} \in D(K): \exists$ extension $(E,V,\nu^2): E = C$. 
If $u \lra \af_1$, then the generic points follow:
\[\begin{array}{lll} x_2 & \raa{\nu} & x_1 \\ \updownarrow && \updownarrow\\
Y = x_2 - \af_2 \ra X = \mathbb{U} + \nu Y\end{array}\]

\item Survive \textbf{Second Descend}: $M^{(2)} = \{(D,U,\nu): \exists$ extension $(E,V,\nu^2): \forall \pp: \exists A_\pp \in E(K_\pp)\}$ which can be effectively determined. 

\item $M = \{(D,U,\nu): \exists A \in D(K)\}$.
\[M \subseteq M^{(2)} \subseteq M^{(1)}\]

\item Goal of the Paper: $\forall \nu^2 \in \Z, Ker(\nu) \subseteq C(K) \exists$ bilinear $T(UU',U'') = T(U,U'') + T(U',U'')$ and skewsymmetric $T(U,U') + T(U',U) = 0$
\[T: M^{(1)} \times M^{(1)} \ra \Q/\Z\]
with kernel $Ker(T) = \{U \in M^{(1)}: T(U,\cdot) = 0\} = M^{(2)}$. 

\item \textbf{SECTION 2}
\item $\forall \delta \in Ker(\nu) \subseteq C(K): \exists \Phi_\delta/K$ (function) with divisor $(\infty)^{-1}(\delta)$.

\item If $x = \nu x_1$ where $x_1$ is another generic point of $C$, then $\Phi_\delta(x_1)^d \in K(x)$ with $div(\Phi_\delta(x_1)) = (\infty)^{-d}(\delta)^d$ where $d$ is the order of $\delta$.

\item If $\delta,\epsilon \in Ker(\nu): \Phi_\delta(x_1 + \epsilon) = \chi(\delta,\epsilon) \Phi_\delta(x_1)$ where $\chi(\delta,\epsilon)^d = 1$ and $\chi(\delta,\epsilon) \in K$. 

\item L1: $a_j$ (any number of them) $\in C(\ACK): \exists g(x/K): \varphi_\delta(x_1):= \frac{\Phi_\delta(x_1)g(\nu x_1)}{g(\nu x_1 - \delta)}$ has neither a pole nor a zero on any $x_1 = a_j$. 

\item L2: $x_1$ generic of $C$; $\delta_i \in Ker(\nu)$; $m_j \in \Z: \sum m_j \delta_j = \infty \Ra \prod (\varphi_{\delta_i}(x_1)^{m_1} \in K(\nu x_1)$. For an alternative independent generic $n_1$: 
\[\theta_\delta(\nu x_1, \nu n_1):= \frac{\varphi_\delta(x_1)\varphi_{\theta}(n_1)}{\varphi_{\theta}x_1 + n_1) \in K(\nu x_1, \nu n_1)\]


\item \textbf{Invariant System}: Assume that $Ker(\nu) = Span_\Z (\epsilon_1,\ldots,\epsilon_s)$ where $\epsilon_i ^{e_i} = \infty = 0_C$. Take $a_1,\ldots, a_s \in K^*$. Define $A_{\epsilon}^e = a$ i.e.: $A_\epsilon = a^{1/e} \in \ACK$ and if $\delta = \sum u_i \epsilon_i \Ra A_\delta = \prod A_{\epsilon_i}^{u_i} = \prod a_i^{u_i/e_i}$. Define $A_\infty = 1$. Notice all orders $e_i$'s are divisors of $\nu^2 \in \Z$. Product of invariant systems are product of the elements defining them, i.e.: $(A\cdot A')_{\epsilon} = a^{1/e}a'^{1/e}$ because the orders $e$ depend on the basis $\epsilon$ not on the invariant system. Equivalence: $\exists$ character $\psi: Aut(\ACK/K) \times Ker(\nu) \ra \mu$ with $\rho(A_\delta) = \psi(\rho,\delta) A_\delta$ where we have previously fixed $\rho \in Aut(\ACK,K)$. 

\item L3: $C(K) \ra$ Invariant Classes: denote $\nu^{-1}\af$ any element in the inverse image. Then $\af \mapsto \{\varphi_\delta(\nu^{-1}\af)\}_\delta$ is a homomorphism independent of $\nu^{-1}\af$, and of the choice $g(x)$ as long as $\varphi^\delta(\nu^{-1}\af)$ is not $0$ or $\infty$. And it has Kernel $\nu C(K)$. 


\item L4: Bijection between Invariant Classes and $\nu$-coverings. 
\item C: The image of the map $\frac{C(K)}{\nu C(K)} \ra Inv.Class \lra \nu-Cover$ is those $(D,U,\nu)$ with a point defined over $K$, i.e.: $\exists p \in D(K)$. 

\end{enumerate}
\end{multicols}

\section{WatkinsStein Tables of EC}
\begin{multicols}{2}
\begin{enumerate}
\item Choose one of the $288$ possibilities for the classes for $c_4,c_6$ modulo $2^63^2,2^63^3$.

\item $c_6 \equiv 3 (4)$ and $c_4$ odd.\\
Or $2^4|c_4$ and $c_6 \equiv 0,8 (32)$ and $c_6 \not \equiv \pm 9 (27)$. \\
Recall $c_4^3 - c_6^2 = 12^3 \Delta \Ra 12^3|c_4^3 - c_6^2$.\\
ABC gives a bound on $c_4$.

\item Since we are working in $Char(\Q) = 0$, then we can reduce our equations to $W(A,B)$ form. Recall that one way of defining the $D$-quadratic twist is 
\[E: y^2 = x^3 + Ax + B \]
\[\Ra E_D: Dy^2 = x^3 + Ax + B\]
\[(Dy,Dx) \Ra E_D: y_2 = x^3 + AD^{-2}x + BD^{-3}\]
Since $c_4 = - A/27$ and $c_6 = -B/54$.
That means $c_{4D} = c_4/D^2$ and $c_{6D}/D^3$. 

\item $E_d$ is the quadratic twists, which means $dY^2 = f(X)$. Assuming that $d$ is square free in $K$, then $(X,Y) \mapsto (X,Y/\sqrt{d})$ is $\cong/K(\sqrt{d})$. Since $\sqrt{d} \notin K$, then it is not $\cong/K$. 

\item $c_4 \leq 1.2^2 \cdot 10^{12} \Ra |\Delta| \leq 10^{12}$.

\item Want Global minimal equation (minimal discriminant) and also minimal discriminant in the quadratic twists.



\item Recall that $p^{12}|\Delta$, $p^4|c_4$, or $p^6|c_6$ means Minimal equation at $p$.\\

\item $\forall p \geq 5: p^6|\Delta$ and $p|c_4 \Ra$ Not minimal for quadratic twists. Replace $E$ by the twist $E_{\tilde{p} = (-1/p)p}$. Recall here that $(-1/p) = (-1)^{(p-1)/2}$. Moreover, if $p|\Delta$, then the conductor at $p$ is defined as: 
\[p^i||N \Ra i = \left\{ \begin{array}{rcl} 1 & \La & p \not| c_4 \\ 2 & \La & p | c_4\end{array}\right.\]

\item $p = 3$: \\
$3^9|c_4^3-c_6^2$ or $3^6|\Delta$, $1 \leq ord_3(c_6) \neq 5:$ replace $E$ by $E_{-3}$ with $(c_4/9, -c_6/27)$. Pf: Always $3^6|\Delta$ and $3|c_6 \Ra 3^2|c_4$ and $3^3|c_6$. Thus $E_{-3}$ does reduces as much as possible the $c_i$'s \\

Moreover, 
if $3\not|c_4$, then good $3 \not|N$ or multiplicative reduction $3||N$.\\
if $3||c_4$ then $$3\not|N$ (good reduction).\\
Otherwise $3^2|c_4$ and 
\[\tilde{c_4} := (c_4/9)\,mod.3 \geq 0\]
\[\tilde{c_6} := (c_6/27)\,mod.9 \geq 0\]
Then Table 1 gives the exponent of $3$ in the conductor. 

\item $p = 2$: Include the prime at infinity (which is the sign) so twisting by $-1$ is part of the analysis.\\
If $2^4|c_4, 2^8|c_6 \Ra$ Replace $E$ by $E_2$ with invariants $(c_4/2^2, c_6/2^3)$ and $\Delta_2 = \Delta/2^6$. \\
If $2^6|c_4,2^6||c_6 \Ra$ Replace $E$ by $\left\begin{array}{c} E_2 \La c_6/8 \equiv 8 (32) \\ E_{-2} \La c_6/8 \equiv 0 (32)$. \\
If $2^4||c_4,2^6||c_6,2^{18}|c_4^3 - c_6^2 \Ra$ Replace $E$ by $\left\begin{array}{c} E_1 \La c_6/64 \equiv 3 (4) \\ E_{-1} \La c_6/64 \equiv 1 (4)$.\\
Finally the power of $2$ of the conductor is given by:
\[2^e||N \Ra e = \left\{\begin{array}{rcl}
2^0 \La 2\not|\Delta, 2\not|c_4; 2|c_4 (\Ra 2^4|c_4), c_6 \equiv 8 (32) \\
2^1 \La 2|\Delta, 2\not|c_4
\end{array}\right.\] 
The rest of cases occur when $2^4|c_4$ and $2^5|c_6$
\[\tilde{c_4} := (c_4/16)\,mod.8 \geq 0\]
\[\tilde{c_6} := (c_6/32)\,mod.8 \geq 0\]
Tables 2 and 3 gives the exponent of $2$ in the conductor.\\

Minimal discriminant at $2$ means minimal discriminant at $2$ unless $2^4||N$ or $2^6||N$. In the first case, disregard the curve. In the second $E_2$ has $N/2$ but $64\Delta$.

\end{enumerate}
\end{multicols}

\section{Watkins ModularDegree}
\begin{multicols}{2}
\begin{enumerate}
\item $L(E,s) = \prod_p(1-\alpha_p p^{-s})^{-1}(1-\beta_pp^{-s})^{-1}$ where if $p\not|N \Ra \beta_p = \overline{\alpha_p}$ and $\alpha_p + \beta_p = Tr(Frob-p)$. If $p||N \Ra \beta_p = 0$ and $\alpha_2^p = 1$. If $p^2|N \Ra \alpha_p = \beta_p = 0$.


\item $L^A(Sym^2E,s) = \prod_p (1-\alpha_p^2 p^{-s})^{-1}\prod_p (1-\alpha_p\beta_p p^{-s})^{-1}(1-\beta_p^2 p^{-s})^{-1}$
Not stable under quadratic twists, convolution of the $L$ function with itself, only needs factors at $p^2|N$ for functional equation. $L^M(Sym^2E,s) = L^A(Sym^2E,s)U(s)$ which is stable under quadratic twists.

\item  
\[\Lambda^M = \left(\frac{\tilde{N}}{4\pi^3}\right)^{s/2}\Gamma(s)\Gamma(s/2) L^m\]
\[L^M(s) = L^M(3-s)\]

\item 
\[\frac{L^A(Sym^2E,s)}{\pi i \Omega} = \frac{deg(\phi)}{Nc^2}\]

\item Recall the notion of the $p$-minimal quadratic twist of $E$ in minimal Weierstrass form. 

\item Say $(F,N_F)$ is the global minimal twist of $E$ and its conductor. 

\item The formulas above indicate $deg(\phi_E) = deg(\phi_F) \cdot \frac{c_E^2}{c_F^2} \cdot \prod_p V_p$
where
\[V_p = \left\{\begin{array}{c} 1 \La E_p = E \\ \frac{\Omega_{E_p}}{\Omega_E} \frac{N_E}{N_{E_p}} L^A_p(E_p)^{-1} \La E_p \neq E \end{array}\]
where every factor is computable, so computing $deg(\phi_F)$ leads to compute $deg(\phi_E)$.

\item For $p \neq 2$ and $E_p \neq E$:
\[V_p = \left\{\begin{array}{c} (p-1)(p+1-Tr(Frob_p))(p+1+tr(Frob_p)) \La p \not|N \\  (p-1)(p+1) \La p||N \\ p \La p^2||N\end{array}\right.\]

\item \textbf{Local Conductors} $\delta_p/ N_F = \prod_p p^{\delta_p}$.\\
Write $\tilde{\delta_p}$ for $\tilde{N} = \prod_p p^{\tilde{\delta_p}}$. Then \\
\[p\not|N_F \Ra \tilde{\delta_p} = 0\] 
\[p||N_F \Ra \tilde{\delta_p} = 1\] 
\[p^2|N_F \Ra U_p(s) = (1\pm p^{1-s})^{-1} \textrm{ or } 1\] 
 
\item $p \geq 5$: \\
\[p \equiv 1 (12); p \equiv 5 (12), p^2|c_6, p||c_4 \Ra U_p(s) = (1 - p^{1-s})^{-1}\]
\[p \equiv 11 (12); p \equiv 7 (12), p^2|c_6, p||c_4 \Ra U_p(s) = (1 + p^{1-s})^{-1}\]

\item $p = 2$: \\
\[2^8|N_F \Ra 2^5||c_4, 2^8|c_6\]
\[2^9|c_6 \Ra U_2(s) = 1 \Ra \tilde{\delta_2} = 4\]
\[2^8||c_6 \Ra \tilde{\delta_2} = 3\]
\[c_4 \equiv 32(128) \ra U_2(s) = (1+2^{1-s})^{-1}\]
\[c_4 \equiv 96(128) \Ra U_2(s) = (1-2^{1-s})^{-1}\]

\item $p = 3$: \\
\[3^4||N_F \Ra c_4 \equiv 9 (27), 3^3||c_6, 3^{-3}c_6 \not \equiv \pm 1(9); \textrm{ or } 3^3||c_4, 3^5||c_6\]
\[3^3||c_6, c_6 \equiv \pm 54(243) \Ra U_3(s) = (1+3^{1-s})^{-1}\]
\[3^3||c_6, c_6 \equiv 108(243) \Ra U_3(s) = (1-3^{1-s})^{-1}\]
\[3^3||c_4, c_4 \equiv 27 (81) \Ra (1-3^{1-s})^{-1}\]
\[3^3||c_4, c_4 \equiv 54 (81) \Ra (1+3^{1-s})^{-1}\]

\item \textbf{Optimal Parametrization}: $\phi: X_0(N) \ra E$ if $\forall \phi': X_0(N) \ra E'$ with $E'$ isogenous to $E$ factors through $\phi$. 
\[\begin{array}{rcl}
X_0(N) & \raa{\phi'}  & E'\\
\phi\downarrow & \nearrow \\
E



\end{array}\] 


\end{document}